\section{Diskussion}

Die Ergebnisse der Webstudie und die durchgeführten Interviewstudien legen nahe, dass eine der Hauptforderungen an Autonome Fahrzeuge und Fahrassistenzsysteme ist, dass diese Systeme besser fahren als Menschen. Aktuelle Unfallstatistiken (vgl. \cite{singh2015critical}) legen nahe, dass dies bereits der Fall ist. Dies wirft die Frage auf, wie man diesen Umstand der breiten Masse kommunizieren kann. Hier gab es eine große Diskrepanz zwischen Web- und Interviewstudie: Während die Interview-Teilnehmer zu einem großen Teil der Meinung waren, dass Fahrassistenzsysteme zumindest in Standardsituationen mindestens so zuverlässig seien wie ein menschlicher Fahrer, ergab sich in den untersuchten Tweets und Kommentaren ein anderes Meinungsbild. Ein Großteil der in der Webstudie gesammelten Kommentare war autonomen Fahrzeugen und Fahrassistenzsystemen äußerst kritisch gegenüber aufgestellt. Unter vielen Kommentatoren herrschte der Konsens, dass ein autonomes Fahrzeug niemals besser fahren könne als sie. Die Befragten der Interviewstudie reagierten deutlich differenzierter und gestanden auch teilweise eigene Unzulänglichkeiten beim Fahren ein, die ein Fahrassistenzsystem ausgleichen könnte oder einem autonomen Fahrzeug aller Voraussicht nach nicht passieren würden.

Diese negative Emotionalisierung in Online-Diskussionen ist ein bekanntes Phänomen. Forscher haben etwa entdeckt, dass Kommentare zum \emph{Brexit} größtenteils von Angst oder Wut bestimmt waren, während positive, aufgeschlossene oder fröhliche Kommentare deutlich seltener gepostet wurden (vgl. \cite[4]{bossetta2018shouting}). Dennoch war die Abstimmung zum Brexit denkbar knapp, da nur die knappe Mehrheit der Briten für den Austritt aus der EU stimmten. Diese Ergebnisse legen nahe, dass negative Emotionen häufiger online zum Ausdruck gebracht werden könnten als positive. Möglicherweise verhält es sich in den Kommentarspalten zu Artikeln, die das autonome Fahren behandeln, ähnlich.