\section{Diskussion}

%Unterschiede Webstudie <--> Interviewstudie
Die Ergebnisse der Webstudie und die durchgeführten Interviewstudien legen nahe, dass eine der Hauptforderungen an autonome Fahrzeuge und Fahrassistenzsysteme ist, dass diese Systeme besser fahren als Menschen. Aktuelle Unfallstatistiken (vgl. \cite{singh2015critical}) legen nahe, dass dies bereits der Fall ist. Dies wirft die Frage auf, wie man diesen Umstand der breiten Masse kommunizieren kann. Hier gab es eine große Diskrepanz zwischen Web- und Interviewstudie: Während die Interview-Teilnehmer zu einem großen Teil der Meinung waren, dass Fahrassistenzsysteme zumindest in Standardsituationen mindestens so zuverlässig seien wie ein menschlicher Fahrer, ergab sich in den untersuchten Tweets und Kommentaren ein anderes Meinungsbild. Ein Großteil der in der Webstudie gesammelten Kommentare war autonomen Fahrzeugen und Fahrassistenzsystemen äußerst kritisch gegenüber aufgestellt. Unter vielen Kommentatoren herrschte der Konsens, dass ein autonomes Fahrzeug niemals besser fahren könne als sie. Die Befragten der Interviewstudie reagierten deutlich differenzierter und gestanden auch teilweise eigene Unzulänglichkeiten beim Fahren ein, die ein Fahrassistenzsystem ausgleichen könnte oder einem autonomen Fahrzeug aller Voraussicht nach nicht passieren würden.

Diese negative Emotionalisierung in Online-Diskussionen ist ein bekanntes Phänomen. Forscher haben etwa entdeckt, dass Kommentare zum \emph{Brexit} größtenteils von Angst oder Wut bestimmt waren, während positive, aufgeschlossene oder fröhliche Kommentare deutlich seltener gepostet wurden (vgl. \cite[4]{bossetta2018shouting}). Dennoch war die Abstimmung zum Brexit denkbar knapp, da nur die knappe Mehrheit der Briten für den Austritt aus der EU stimmten. Diese Ergebnisse legen nahe, dass negative Emotionen häufiger online zum Ausdruck gebracht werden könnten als positive. Möglicherweise verhält es sich in den Kommentarspalten zu Artikeln, die das autonome Fahren behandeln, ähnlich.

%unerwartetes Verhalten
Ein weiterer von vielen Befragten angesprochener Punkt war, dass autonome Fahrzeuge und Fahrzeuge mit Fahrassistenzsystemen kein unerwartetes oder unübliches Verhalten an den Tag legen sollen. Hier wäre weitere Forschung notwendig um zu bestimmen, wie groß eine etwaige Abweichung vom Fahrverhalten eines durchschnittlichen Autofahrers bei solchen Fahrzeugen sein dürfte. Einige Hersteller Autonomer Fahrzeuge rühmen sich beispielsweise damit, durch das Fahrverhalten Sprit zu sparen. Hier müsste eine Abschätzung geschehen, ob dieses (teilweise für einige ungeowhnte) Fahrverhalten dennoch von der breiten Masse akzeptiert werden würde oder ob die Hersteller einige Abstriche machen sollten, um etwas weniger Sprit zu sparen und dafür mehr Akzeptanz zu gewinnen.

%Autobahnfahrten
Beinahe alle Befragten können sich vorstellen, Autobahnfahrten vollständig an ihr Auto abzugeben. Solche Autobahnfahrten sind oft eintönig und monoton. Solche Aufgaben fallen Menschen oft sehr schwer, Computern dafür umso leichter (vgl. \cite[15]{norman2013design}). An dieser Stelle könnten Hersteller noch stärker ansetzen, um die Vorteile und die höhere Sicherheit von autonomen Fahrzeugen zu unterstreichen. Denn während Menschen gerade auf längeren Autobahnfahrten fast unweigerlich erschöpft werden und sich nicht länger auf den Straßenverkehr konzentrieren können (vgl. \cite{ting2008driver}), erfahren Computersysteme solche Ermüdungseffekte nicht. An dieser Stelle kann also eine der großen Stärken von autonomen Fahrzeugen einer großen Schwäche von Menschen gegenübergestellt werden. Dies könnte eine gute Möglichkeit für vertrauensbildende Maßnahmen darstellen.

Gleichzeitig ging aus der Webstudie aber auch hervor, dass die Freude am Fahren ein nicht zu unterschätzender Faktor ist. Hier entsteht ein Spannungsfeld vom Autofahren als Hobby vs.\, Autofahren als Fortbewegungsmittel. Auch mit guten und erfolgreichen vertrauensbildenden Maßnahmen kann es schwierig werden, solche \glq Hobby-Fahrer\grq{} vom autonomen Fahren zu überzeugen. In der Zukunftsvision des verbundenen Straßenverkehrs (vgl. \cite{gerla2014internet}) ist es jedoch zwingend notwendig, dass \emph{alle} Autos miteinander verbunden sind. Und auch in der Interviewstudie wurde von einigen Teilnehmern angemerkt, dass zu einem großen Teil rücksichtslose Autofahrer als größtes Hindernis für zuverlässig funktionierendes, autonomes Fahren angesehen werden:

\begin{quote}
  Und ich glaube, wenn irgendwann komplett mit autonomen Fahren ist und keiner mehr manuell fährt, dass es dann… dass es dann deutlich sicherer ist als jetzt. Aber bis dahin würde ich nie hundertprozentig sicher sein. (B01, Z. 52)
\end{quote}

Dies wirft eine der spannenden ethisch/moralischen Fragen in der Zukunft auf: Darf man Leuten ihr Hobby wegnehmen, um den Straßenverkehr insgesamt sicherer zu gestalten? Ist dies ein zu tiefer Eingriff in die Persönlichkeitsrechte? Diese Fragen können nur im gesellschaftlichen Diskurs gelöst werden (vgl. \cite[12]{renn2010risk}).

Neben solchen gesamtgesellschaftlichen Problemstellungen sind in den Interviewstudien aber auch Aspekte genannt worden, die von Herstellern autonomer Fahrzeuge verhältnismäßig einfach angegangen werden könnten. Die meisten der Befragten hatten bislang beinahe keinerlei Berührungspunkte mit autonomen Fahrzeugen und wünschten sich leichtere Möglichkeiten zu Testfahrten. Hier könnten Hersteller, beispielsweise mit Aktionstagen, unkompliziert Möglichkeiten zum Kennenlernen der Technologie anbieten. Die Befragten waren sich auch einig, dass eine transparente Kommunikation von Seiten der Hersteller wichtig ist, damit potenzielle Nutzer sich selbst ein Bild machen und Vertrauen zu autonomen Fahrzeugen fassen können. Unter anderem wurde gefordert, dass auch etwaige Mängel und Schwächen des Systems ausdrücklich kommuniziert werden sollten.
