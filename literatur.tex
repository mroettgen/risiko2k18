\section{Literaturgestützte Einführung}

\subsection{Risiko und Sicherheit}
Sicherheit ist ein Begriff, für den beinahe jedes Berufsfeld eine eigene Nuancierung hat. Dadurch unterscheidet sich die Definition von Sicherheit zwischen verschiedenen Disziplinen und Kontexten. Im folgenden Abschnitt soll ein kurzer Überblick über unterschiedliche Definitionen von Sicherheit gegeben werden.

\subsubsection*{Sicherheit und Akzeptanz}

Rothkegel definiert vier verschiedene Sicherheits-Modelle, die bei der Sicherheitskommunikation genutzt werden können, um \glqq \glq gelingende\grq \, bzw. \glq gedeihliche\grq \, Kommunikation zum Thema Sicherheit\grqq \, herzustellen (\cite[125]{rothkegel2013sicherheitskommunikation}):
\begin{itemize}
  \item Sicherheit als Abwesenheit von Risiko und Gefahr
  \item Sicherheit als Umgang mit und Steuerung von Risiken
  \item Sicherheit als Umgang mit und Steuerung von Gefahren und
  \item Sicherheit als (interner) Selbstschutz
\end{itemize}

Bei der Definition von Sicherheit als Umgang mit und Steuerung von Risiken wird oft eine Einordnung des Risikos durch Berechnung der Auftretenswahrscheinlichkeit des Schadensausmaßes vorgenommen (vgl. ebd.). Aufgrund dieses Kalküls wird anschließend entschieden, ob man beispielsweise als Hersteller oder Verbraucher gewillt ist, das Risiko einzugehen.

Bei der Definition von Sicherheit als Umgang mit und Steuerung von Gefahren geht es darum, die Gefahr zu minimieren und einzudämmen. Hier wird zwischen aktiver und passiver Sicherheit unterscheiden. Aktive Sicherheit wehrt die Gefahr ab -- etwa durch ABS-Systeme in einem Auto, die das Bremsen sicherer gestalten. Passive Sicherheit schützt dagegen vor dem Schadenseintritt. Ein Beispiel wären Airbags, die einen Autounfall nicht verhindern, den Fahrer im Falle eines solchen Unfalls aber schützen (vgl. ebd., S. 130).

Rothkegel kommt in ihrer Arbeit zu dem Schluss, dass in von Herstellern ausgelösten Kommunikationssituationen \glqq kein Raum für das Thema Sicherheit\grqq \, sei (\cite[132]{rothkegel2013sicherheitskommunikation}). Stattdessen fokussiert eine solche Kommunikation auf das Erlebnis sowie auf Wunschdenken. Wird doch in der Öffentlichkeit über Risiken gesprochen, kommt es zu einer Mischung von Fach- und Alltagswissen und von Fach- und Alltagssprache, was in Kommunikationsproblemen resultiert (vgl. \cite[134]{rothkegel2013sicherheitskommunikation}). Rothkegel schließt daraus, dass \glqq [e]ine Kommunikationskultur, in der der Begriff des Risikos die Sicherheitskommunikation bestimmt, [...] per se konflikthaft [ist]\grqq \, (ebd., S. 135).

Laut Banse nimmt die allgemeine Risikoakzeptanz gesellschaftsweit ab, während das Sicherheitsverlangen im gleichen Maße zunimmt (vgl. \cite[3]{banse2018technik}). Dieses Verlangen nach Sicherheit kann dabei nicht nur Statistiken, Zahlen, Daten und Fakten befriedigt werden. Das liegt daran, dass Sicherheit nicht lediglich aus rationalem Wissen entstehe, sondern auch aus \glqq einem intuitiven Verständnis, aus Erfahrungen und Erwartungen, aus Hoffnungen und Ängsten, aus erlebten Mitgestaltungsmöglichkeiten bei technischen Problemlösungsprozessen oder zumindest wahrgenommenen Eingriffsmöglichkeiten in technische Abläufe bzw. aus Ohnmachtsgefühlen angesichts einer scheinbaren Eigendynamik des Technischen\grqq \, (\cite[4]{banse2018technik}).

Genau eine solche Eigendynamik scheinen dabei immer mehr Systeme zu entwickeln. Viele technische Geräte scheinen selbst für versierte Nutzer unberechenbar zu sein (vgl. \cite[5]{norman2013design}), was zu Frustration und Skepsis gegenüber solche Systeme führt. Dazu kommt, dass bei vielen komplexen technologischen Lösungen eine Begrenzung der Folgen fast nicht durchzuführen ist und auch genaues Wissen über Schadensausmaß und Eintrittswahrscheinlichkeit kaum zu ermitteln ist (vgl. \cite[12]{banse2018technik}).

\subsubsection*{Risiken}

Einen Wandel des Risikobegriffs in den vergangenen Jahrhunderten bemerkt Lau (1989). Er stellt fest, dass sich das gestiegene Gefahrenbewusstsein vor allem gegen Konsequenzen technologischer Entwicklungen wendet (vgl. \cite[418]{lau1989risikodiskurse}). Einen Grund dafür sieht er darin, dass sich der Begriff des Risikos geändert hat. Früher waren Risiken \glqq Experimente mit der eigenen Person\grqq \, waren (vgl. ebd., S. 421), etwa wenn ein Seemann nach Indien aufbrach und damit bewusst ein Risiko einging. Epidemien, Unfälle, Kriege und Ähnliches wurden dagegen als \glqq allgemeine irdische Lebensgefährdungen\grqq \, aufgefasst (vgl. ebd.).

Dies änderte sich mit dem Aufkommen des Versicherungswesens. Risiken wurden berechnet und quantifiziert. Sie waren nicht länger ein \glq Experiment\grq , das im Falle eines Erfolgs Ruhm, Reichtum oder Erfahrung mit sich brachte. Vielmehr handelte es um Ereignisse wie etwa das Abbrennen des eigenen Wohnhauses, das von der Versicherung gedeckt wurde (vgl. ebd., S. 422).

Heutige, durch Technologie ausgelöste Risiken stellen laut Lau eine Mischform dieser beiden Risikoverständnisse dar. Sie werden zwar nicht freiwillig eingegangen, haben ihre Ursachen aber im Entscheiden und Handeln von Personen oder Institutionen (vgl. ebd., S. 423). Damit sind solche Risiken gleichzeitig auf menschliches Handeln zurückzuführen und haben für Betroffene die gleiche Anonymität wie etwa Naturkatastrophen, wodurch sie laut Lau eine \glqq soziale Sprengkraft\grqq \, entfalten (vgl. ebd., S. 433). Eine Möglichkeit, solchen Risiken diese soziale Sprengkraft zu nehmen ist es, solche technologisch erzeugten Risiken zu \glq natürlichen Gefahren\grq \, umzudefinieren (vgl. ebd.), also zu zeigen, dass das durch die Technologie ausgelöste Risiko vergleichbar (oder sogar besser) ist als ein ähnliches, natürliches Risiko.



\subsection{Risikokommunikation}
Kasperson (2014) fasst zusammen, dass in den letzten 30 Jahren zwar viel theoretische und wissenschaftliche Arbeit auf dem Feld der Risikokommunikation geleistet wurde, sich in der Praxis bislang aber kaum etwas geändert habe (vgl. \cite[1234]{kasperson2014four}). Er stellt vier Prinzipien auf, die die Risikokommunikation erfolgreicher gestalten sollen (vgl. ebd., S. 1237f.):
\begin{enumerate}
  \item Der Risikokommunikation eines Projektes muss wegen der gestiegenen Anforderungen in der Bevölkerung mehr Mittel zur Verfügung gestellt bekommen und ambitioniertere Ziele verfolgen als dies bisher häufig der Fall ist
  \item Die Kommunikation selbst soll nicht nur auf Expertenebene erfolgen, sondern Konsumenten auch in ihrem täglichen Leben erreichen
  \item Je größer die Ungewissheiten betreffend eines Risikos sind, desto mehr muss kommuniziert werden. Außerdem muss kommuniziert werden, welche Risiken sich in welchem Zeitraum senken lassen können
  \item Ziele, Struktur und Durchführung der Risikokommunikation müssen dem bestehenden sozialen Mistrauen angepasst werden. Dabei muss darauf eingegangen werden, dass innerhalb der Bevölkerung das Vertrauen in Institutionen in den letzten Jahren erheblich geschrumpft ist (vgl. ebd., S. 1236)
\end{enumerate}

Sehr ähnliche Anforderungen stellt auch Renn (2010) auf. Er definiert als Aufgabe der Risikokommunikation, Menschen mit genügend Wissen auszustatten, damit sie selbstständig fundierte Entscheidungen treffen können (vgl. \cite[81]{renn2010risk}). Da laut ihm die Anforderungen an Risikokommunikateure in den vergangen Jahren gewachsen seien, müssten Unternehmen und Regierungen der Öffentlichkeit heute mehr Informationen zur Verfügung stellen als früher (vgl. ebd., S. 82ff.). Dabei gibt es drei Ebenen, die bei solchen \glq Risiko-Debatten\grq \, beachtet werden müssen:
\begin{itemize}
  \item \textbf{Fakten und Wahrscheinlichkeiten:} Risikokommunikateure müssen der Öffentlichkeit nicht nur Zahlen, Daten und Fakten präsentieren, sondern auch dabei unterstützen diese selbstständig interpretieren zu können
  \item \textbf{Expertise, Erfahrung und Leistung des Unternehmens:} Hierfür muss ein Dialog zwischen den Stakeholdern und der Öffentlichkeit hergestellt werden
  \item \textbf{Konflikte mit bestehenden, persönlichen Wertesystemen und Erfahrungen:} Diese Ebene ist laut Renn die am schwierigsten zu erreichende. Der Umgang mit persönlichen Wertevorstellungen und Erfahrungen erfordert ein hohes Maß an Empathie und erfordert eine erhöhte Nahbarkeit der Stakeholder
\end{itemize}

Insgesamt fasst Renn zusammen, dass Risikokommunikation über bloße PR-Arbeit und Informationen hinausgeht und die Bürger auch auf einer persönlichen Ebene erreichen muss (vgl. ebd., S. 95).

\subsection{Autonomes Fahren}

Autonome Fahrzeuge werden in den kommenden Jahren aller Voraussicht nach ein integraler Bestandteil des Straßenverkehrs werden. Neben gesteigertem Fahrkomfort können solche selbstfahrenden Autos auch die Sicherheit im Straßenverkehr positiv beeinflussen – unter anderem, indem Unfälle durch menschliches Versagen verhindert werden und ältere und behinderte Menschen in ihrer Mobilität unterstützt werden können (vgl.  \cite[167]{fagnant2015preparing}).

Tatsächlich zeigte eine Studie der National Highway Traffic Safety Administration, dass 90\% der Unfälle, die in der Unfallstatistik gelistet wurden, durch menschliches Versagen verursacht wurden. 40\% der Unfälle mit tödlichem Ausgang waren auf Alkohol- oder Drogenkonsum, Ablenkung oder Müdigkeit zurückzuführen (vgl. \cite{singh2015critical}). Hier könnten selbstfahrende Fahrzeuge für eine erhöhte Sicherheit für alle Straßenverkehrsteilnehmer sorgen: \glqq Self-driven Self-driven vehicles would not fall prey to human failings, suggesting the potential for at least a 40\% fatal crash-rate reduction, assuming automated malfunctions are minimal and everything else remains constant. Such reductions do not reflect crashes due to speeding, aggressive driving, over-compensation, inexperience, slow reaction times, inattention and various other driver shortcomings.\grqq \ (\cite[169]{fagnant2015preparing}).

In der Zukunft werden Autos höchstwahrscheinlich miteinander vernetzt sein und stets miteinander kommunizieren. Dies wird die Notwendigkeit eines menschlichen Fahrers weiter minimieren (vgl. \cite[241]{gerla2014internet}). Mit diesem Schritt weg vom individuellen Vehikel und hin zu einer stets über die \emph{Cloud} autonomen Fahrzeugflotte kann die Sicherheit im Straßenverkehr noch weiter erhöht werden (vgl. ebd.).

Doch obwohl diese neue Technologie rein statistisch betrachtet sicherer zu sein scheint, herrscht viel Skepsis gegenüber dem autonomen Fahren. Dies liegt unter daran, dass es gerade bei neuartigen Technologien oft eine Diskrepanz zwischen tatsächlichem und wahrgenommenem Risiko besteht (vgl. \cite[106]{hengstler2016applied}). Eine Möglichkeit, dieses wahrgenommene Risiko zu verkleinern, ist, Vertrauen in das Produkt zu schaffen (vgl. \cite{rousseau1998not}). Doch nicht nur das Vertrauen in die Technologie an sich ist ein entscheidender Faktor zur allgemeinen Akzeptanz, auch das Vertrauen in den \emph{Hersteller} ist wichtig (vgl. \cite[107]{hengstler2016applied}). Hengstler et al. kommen dabei zu dem Schluss, dass eine gute Kommunikation seitens des Herstellers positive Effekte auf das Vertrauen in den Hersteller und damit auch auf das Vertrauen in die neue Technologie haben kann (ebd.).
