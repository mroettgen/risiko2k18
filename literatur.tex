\section{Literaturgestützte Einführung}

\subsection{Definition Risiko}

\subsection{Definition Sicherheit}
Sicherheit ist ein Begriff, für den beinahe jedes Berufsfeld eine eigene Nuancierung hat. Dadurch unterscheidet sich die Definition von Sicherheit zwischen verschiedenen Disziplinen und Kontexten. Im folgenden Abschnitt soll ein kurzer Überblick über unterschiedliche Definitionen von Sicherheit gegeben werden.

\subsubsection*{Sicherheit und Akzeptanz}

Rothkegel definiert vier verschiedene Sicherheits-Modelle, die bei der Sicherheitskommunikation genutzt werden können, um \glqq \glq gelingende\grq \, bzw. \glq gedeihliche\grq \, Kommunikation zum Thema Sicherheit\grqq \, herzustellen (\cite[125]{rothkegel2013sicherheitskommunikation}):
\begin{itemize}
  \item Sicherheit als Abwesenheit von Risiko und Gefahr
  \item Sicherheit als Umgang mit und Steuerung von Risiken
  \item Sicherheit als Umgang mit und Steuerung von Gefahren und
  \item Sicherheit als (interner) Selbstschutz
\end{itemize}

Bei der Definition von Sicherheit als Umgang mit und Steuerung von Risiken wird oft eine Einordnung des Risikos durch Berechnung der Auftretenswahrscheinlichkeit des Schadensausmaßes vorgenommen (vgl. ebd.). Aufgrund dieses Kalküls wird anschließend entschieden, ob man beispielsweise als Hersteller oder Verbraucher gewillt ist, das Risiko einzugehen.

Bei der Definition von Sicherheit als Umgang mit und Steuerung von Gefahren geht es darum, die Gefahr zu minimieren und einzudämmen. Hier wird zwischen aktiver und passiver Sicherheit unterscheiden. Aktive Sicherheit wehrt die Gefahr ab -- etwa durch ABS-Systeme in einem Auto, die das Bremsen sicherer gestalten. Passive Sicherheit schützt dagegen vor dem Schadenseintritt. Ein Beispiel wären Airbags, die einen Autounfall nicht verhindern, den Fahrer im Falle eines solchen Unfalls aber schützen (vgl. ebd., S. 130).

Rothkegel kommt in ihrer Arbeit zu dem Schluss, dass in von Herstellern ausgelösten Kommunikationssituationen \glqq kein Raum für das Thema Sicherheit\grqq \, sei (\cite[132]{rothkegel2013sicherheitskommunikation}). Stattdessen fokussiert eine solche Kommunikation auf das Erlebnis sowie auf Wunschdenken. Wird doch in der Öffentlichkeit über Risiken gesprochen, kommt es zu einer Mischung von Fach- und Alltagswissen und von Fach- und Alltagssprache, was in Kommunikationsproblemen resultiert (vgl. \cite[134]{rothkegel2013sicherheitskommunikation}). Rothkegel schließt daraus, dass \glqq [e]ine Kommunikationskultur, in der der Begriff des Risikos die Sicherheitskommunikation bestimmt, [...] per se konflikthaft [ist] \grqq \, (ebd., S. 135).

Laut Banse nimmt die allgemeine Risikoakzeptanz gesellschaftsweit ab, während das Sicherheitsverlangen im gleichen Maße zunimmt (vgl. \cite[3]{banse2018technik}). Dieses Verlangen nach Sicherheit kann dabei nicht nur Statistiken, Zahlen, Daten und Fakten befriedigt werden. Das liegt daran, dass Sicherheit nicht lediglich aus rationalem Wissen entstehe, sondern auch aus \glqq einem intuitiven Verständnis, aus Erfahrungen und Erwartungen, aus Hoffnungen und Ängsten, aus erlebten Mitgestaltungsmöglichkeiten bei technischen Problemlösungsprozessen oder zumindest wahrgenommenen Eingriffsmöglichkeiten in technische Abläufe bzw. aus Ohnmachtsgefühlen angesichts einer scheinbaren Eigendynamik des Technischen\grqq (\cite[4]{banse2018technik}).

Genau eine solche Eigendynamik scheinen dabei immer mehr Systeme zu entwickeln. Viele technische Geräte scheinen selbst für versierte Nutzer unberechenbar zu sein (vgl. \cite[5]{norman2013design}), was zu Frustration und Skepsis gegenüber solche Systeme führt. Dazu kommt, dass bei vielen komplexen technologischen Lösungen eine Begrenzung der Folgen fast nicht durchzuführen ist und auch genaues Wissen über Schadensausmaß und Eintrittswahrscheinlichkeit kaum zu ermitteln ist (vgl. \cite[12]{banse2018technik}).

\subsection{Wahrnehmung von Sicherheit}
blablabla hier ist Text

\subsection{Autonomes Fahren}

Autonome Fahrzeuge werden in den kommenden Jahren aller Voraussicht nach ein integraler Bestandteil des Straßenverkehrs werden. Neben gesteigertem Fahrkomfort können solche selbstfahrenden Autos auch die Sicherheit im Straßenverkehr positiv beeinflussen – unter anderem, indem Unfälle durch menschliches Versagen verhindert werden und ältere und behinderte Menschen in ihrer Mobilität unterstützt werden können (vgl.  \cite[167]{fagnant2015preparing}).

Tatsächlich zeigte eine Studie der National Highway Traffic Safety Administration, dass 90\% der Unfälle, die in der Unfallstatistik gelistet wurden, durch menschliches Versagen verursacht wurden. 40\% der Unfälle mit tödlichem Ausgang waren auf Alkohol- oder Drogenkonsum, Ablenkung oder Müdigkeit zurückzuführen (vgl. \cite{singh2015critical}). Hier könnten selbstfahrende Fahrzeuge für eine erhöhte Sicherheit für alle Straßenverkehrsteilnehmer sorgen: \glqq Self-driven Self-driven vehicles would not fall prey to human failings, suggesting the potential for at least a 40\% fatal crash-rate reduction, assuming automated malfunctions are minimal and everything else remains constant. Such reductions do not reflect crashes due to speeding, aggressive driving, over-compensation, inexperience, slow reaction times, inattention and various other driver shortcomings.\grqq \ (\cite[169]{fagnant2015preparing}).

In der Zukunft werden Autos höchstwahrscheinlich miteinander vernetzt sein und stets miteinander kommunizieren. Dies wird die Notwendigkeit eines menschlichen Fahrers weiter minimieren (vgl. \cite[241]{gerla2014internet}). Mit diesem Schritt weg vom individuellen Vehikel und hin zu einer stets über die \emph{Cloud} autonomen Fahrzeugflotte kann die Sicherheit im Straßenverkehr noch weiter erhöht werden (vgl. ebd.).

Doch obwohl diese neue Technologie rein statistisch betrachtet sicherer zu sein scheint, herrscht viel Skepsis gegenüber dem autonomen Fahren. Dies liegt unter daran, dass es gerade bei neuartigen Technologien oft eine Diskrepanz zwischen tatsächlichem und wahrgenommenem Risiko besteht (vgl. \cite[106]{hengstler2016applied}). Eine Möglichkeit, dieses wahrgenommene Risiko zu verkleinern, ist, Vertrauen in das Produkt zu schaffen (vgl. \cite{rousseau1998not}). Doch nicht nur das Vertrauen in die Technologie an sich ist ein entscheidender Faktor zur allgemeinen Akzeptanz, auch das Vertrauen in den \emph{Hersteller} ist wichtig (vgl. \cite[107]{hengstler2016applied}). Hengstler et al. kommen dabei zu dem Schluss, dass eine gute Kommunikation seitens des Herstellers positive Effekte auf das Vertrauen in den Hersteller und damit auch auf das Vertrauen in die neue Technologie haben kann (ebd.).
