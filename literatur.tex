\section{Literaturgestützte Einführung}
Autonome Fahrzeuge werden in den kommenden Jahren aller Voraussicht nach ein integraler Bestandteil des Straßenverkehrs werden. Neben gesteigertem Fahrkomfort können solche selbstfahrenden Autos auch die Sicherheit im Straßenverkehr positiv beeinflussen – unter anderem, indem Unfälle durch menschliches Versagen verhindert werden und ältere und behinderte Menschen in ihrer Mobilität unterstützt werden können (vgl.  \cite[167]{fagnant2015preparing}).

Tatsächlich zeigte eine Studie der National Highway Traffic Safety Administration, dass 90\% der Unfälle, die in der Unfallstatistik gelistet wurden, durch menschliches Versagen verursacht wurden. 40\% der Unfälle mit tödlichem Ausgang waren auf Alkohol- oder Drogenkonsum, Ablenkung oder Müdigkeit zurückzuführen (vgl. \cite{singh2015critical}). Hier könnten selbstfahrende Fahrzeuge für eine erhöhte Sicherheit für alle Straßenverkehrsteilnehmer sorgen: \glqq Self-driven Self-driven vehicles would not fall prey to human failings, suggesting the potential for at least a 40\% fatal crash-rate reduction, assuming automated malfunctions are minimal and everything else remains constant. Such reductions do not reflect crashes due to speeding, aggressive driving, over-compensation, inexperience, slow reaction times, inattention and various other driver shortcomings.\grqq \ (\cite[169]{fagnant2015preparing}).

In der Zukunft werden Autos höchstwahrscheinlich miteinander vernetzt sein und stets miteinander kommunizieren. Dies wird die Notwendigkeit eines menschlichen Fahrers weiter minimieren (vgl. \cite[241]{gerla2014internet}). Mit diesem Schritt weg vom individuellen Vehikel und hin zu einer stets über die \emph{Cloud} autonomen Fahrzeugflotte kann die Sicherheit im Straßenverkehr noch weiter erhöht werden (vgl. ebd.).

Doch obwohl diese neue Technologie rein statistisch betrachtet sicherer zu sein scheint, herrscht viel Skepsis gegenüber dem autonomen Fahren. Dies liegt unter daran, dass es gerade bei neuartigen Technologien oft eine Diskrepanz zwischen tatsächlichem und wahrgenommenem Risiko besteht (vgl. \cite[106]{hengstler2016applied}). Eine Möglichkeit, dieses wahrgenommene Risiko zu verkleinern, ist, Vertrauen in das Produkt zu schaffen (vgl. \cite{rousseau1998not}). Doch nicht nur das Vertrauen in die Technologie an sich ist ein entscheidender Faktor zur allgemeinen Akzeptanz, auch das Vertrauen in den \emph{Hersteller} ist wichtig (vgl. \cite[107]{hengstler2016applied}). Hengstler et al. kommen dabei zu dem Schluss, dass eine gute Kommunikation seitens des Herstellers positive Effekte auf das Vertrauen in den Hersteller und damit auch auf das Vertrauen in die neue Technologie haben kann (ebd.).
