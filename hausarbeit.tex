\documentclass[12pt]{article}
\usepackage[utf8]{inputenc}
\usepackage[T1]{fontenc}
\usepackage[german]{babel}


\usepackage{graphicx}

%   Abbildungen müssen in einem Ordner "Abbildungen" liegen
\graphicspath{ {Abbildungen/} }
\usepackage{pdfpages}
\usepackage{csquotes}

\usepackage{etoolbox}
\AtBeginEnvironment{quote}{\small\setstretch{.25}}

\usepackage{helvet}
\renewcommand{\familydefault}{\sfdefault}
\renewcommand\labelenumi{(\theenumi)}

\usepackage[style=authoryear-icomp,maxcitenames=2,backend=biber]{biblatex}
\addbibresource{bibliography.bib}

\usepackage{titlesec}
\titlespacing\section{0pt}{12pt plus 4pt minus 2pt}{0pt plus 2pt minus 2pt}
\titlespacing\subsection{0pt}{12pt plus 4pt minus 2pt}{0pt plus 2pt minus 2pt}
\titlespacing\subsubsection{0pt}{12pt plus 4pt minus 2pt}{0pt plus 2pt minus 2pt}

\usepackage{setspace}
\usepackage{color}
\definecolor{hgray}{gray}{0.5}

\usepackage{enumitem}
\setlist[itemize]{noitemsep, topsep=0pt}
\setlist[enumerate]{noitemsep, topsep=0pt}

\usepackage[hidelinks,
pdfpagelabels,
pdfstartview = FitH,
bookmarksopen = true,
bookmarksnumbered = true,
linkcolor = black,
plainpages = false,
hypertexnames = false,
citecolor = black] {hyperref}


\usepackage{geometry}
\geometry{
  left=2.5cm,
  right=2.5cm,
  top=2.5cm,
  bottom=2cm,
  bindingoffset=0mm
}


\makeatletter
\title{Aspekte der Wahrnehmung von Sicherheit bei der Fahrer-Fahrzeug-Interaktion}\let\Title\@title     %   Titel der Arbeit eintragen
\makeatother



\usepackage{fancyhdr}
\usepackage{afterpage}
\fancypagestyle{MRstyle}{
    \fancyhf{}
    \fancyhead[L]{\textit{\textcolor{hgray}{\Title}}}
    \fancyfoot[c]{\textcolor{hgray}{\thepage}}
}

\fancypagestyle{appendix}{
    \fancyhf{}
    \fancyhead[L]{\textit{\textcolor{hgray}{\Title}}}
    \fancyhead[R]{\textcolor{hgray}{\rightmark}}
    \fancyfoot[c]{\textcolor{hgray}{\thepage}}
}

\usepackage[hang]{footmisc}

\newcommand\fakesection[1]{%
    \markboth{#1}{#1}}

\usepackage{parskip}
\DefineBibliographyStrings{german}{
   andothers = {{et\,al\adddot}},
}


% Das Dokument geht hier los:

\begin{document}
\pagestyle{MRstyle}

\setstretch{1.15}

\begin{titlepage}

    \large
    RWTH Aachen\\
    Institut für Sprach- und Kommunikationswissenschaft\\
    Professur für Textlinguistik und Technikkommunikation\\
    Prof. Dr. E.-M. Jakobs

    \vspace{5cm}
    \Large
    \doublespacing{
        \textit{Hausarbeit zum Seminar Risikokommunikation\\}
        \textbf{\Title}
    }

    \vspace{7cm}
    \normalsize
    \setstretch{1.2}
    vorgelegt von:\\
    Maximilian Röttgen (Mat.-Nr.: 332048)\\ % Hier Name, Matr. Nr. etc. einfügen
    Martin Schmitz (Mat.-Nr.: 320669)\\
    Joshua Olbrich (Mat.-Nr.: ??????)

    \vfill

    Aachen, \today
\afterpage{\cfoot{\textcolor{hgray}{\thepage}}}

\end{titlepage}


\pagenumbering{Roman}

\tableofcontents
\clearpage
\pagenumbering{gobble}
\section*{Zusammenfassung}

\clearpage
\pagenumbering{arabic}

\section{Einleitung}

\clearpage
\section{Literaturgestützte Einführung}

\clearpage
\section{Methodik}

\subsection{Web-Studie}

\clearpage
\section{Ergebnisse}
Im Folgenden werden die Ergebnisse der Studie vorgestellt. Da die Bewertung der Sicherheit vor den jeweilig persönlichen Hintergründen der Befragten zu betrachten sind, werden zunächst einige Vorinformationen über die Befragten betrachtet.
\subsection*{Personendaten}
Unter den sechs Befragten waren fünf Männer und eine Frau. Alle Befragten waren zwischen 23 und 29 Jahren alt, mit einem mittleren Alter von 25 Jahren, Median von 24 Jahren. Die Befragten hatten ihren Führerschein seit mindestens fünf Jahren. Der erfahrenste Fahrer war bereits 12 Jahre im Besitz der Fahrerlaubnis.

Die Hälfte der Befragten war zum Zeitpunkt der Erhebung im Besitz eines eigenen PKW, die anderen 38\,\% nutzten geteilte Fahrzeuge, z.\,B. der Eltern oder Car-Sharing Angebote.

Familienstand und Einkommen wurden nicht erhoben, die Mehrzahl der Befragten ist jedoch als Student tätig.

\subsection{Kenntnisstand der Befragten}
Der Wissensstand der Befragten zum Thema Fahrassistenzsysteme und autonomes Fahren war zum Zeitpunkt der Befragung gemischt. Auch, wenn die Mehrheit der Befragten selbst noch keine praktische Erfahrung mit entsprechenden Systemen gemacht hat, waren die Befragten durch die Presse (B03, Z. 98; B04, Z. 113) und durch allgemeines Vorwissen (B01, Z. 74ff) zumindest grundsätzlich über das Thema informiert.

Detail- und Anwendungswissen war bei der Hälfte der Befragten nicht vorhanden (B02, Z.44; B03, Z. 30, B05, Z. 24), die restlichen Befragten können  zu einzelnen Assistenzsystemen aus eigenen Erfahrungen berichten (B01, Z. 34, 38; B04 Z. 32-37; B06, Z. 12).

Konkrete Kenntnisse haben die Kandidaten zu folgenden Assistenzsystemen:
\begin{itemize}
    \item Bremsassistent (B01, Z. 34; B04, Z. 29)
    \item Verkehrszeichenerkennung (B01, Z. 38)
    \item Spurhalteassistent (B01, Z. 38; B04, Z. 29)
    \item Abstandshalter (B04, Z. 29)
    \item Tempomat (B04, Z. 29)
    \item Einparkhilfen (B03, Z. 32; B06, Z. 12)
\end{itemize}

Kandidaten, die keine Erfahrungen mit Fahrassistenzsystemen oder autonomen Fahrzeugen machen konnten, nannten dafür folgende Gründe:
\begin{itemize}
    \item Kein Zugang zu entsprechenden Fahrzeugen (B03, Z. 32, 78; B06, Z. 12)
    \item Eigener PKW nicht mit Fahrassistenzsystemen ausgestattet (B02, Z. 46)
    \item Fahrassistenzsysteme kein Kaufkriterium (B06, Z. 18)
\end{itemize}

\subsection{Vorannahmen der Befragten}
Einige Äußerungen der Kandidaten ließen auf Vorannahmen bezüglich autonomem und unterstütztem Fahren schließen. Im Folgenden sind diese Vorannahmen aufgelistet und erläutert. Diese Vorannahmen dienen als Kontext zur besseren Einordnung der Ergebnisse.
\subsubsection*{Berichterstattung in der Presse}
Ein Befragter äußerte, dass er von einer negativen Färbung der Berichterstattung in den Medien über das Thema autonomes/assistiertes Fahren ausgeht.
\begin{quote}
    Meistens kommt ja auch nur das Negative dann in die Presse und nicht das Auto fährt jetzt seit 100.000 Km autonom erfolgreich. Nein, da kommt was, wenn es einen übergebretzelt hat. (B04, Z. 113)
\end{quote}

\subsubsection*{Überlegenheit von Technik}
Es gab weiterhin Äußerungen der Befragten, die nahelegen, dass Grundannahmen darüber vorherrschen, ob und wenn ja in welchen Fällen die Technologie besser ist als der Mensch.
\begin{quote}
    Wenn [...] wirklich alle Autos miteinander vernetzt wären, dann hättest du wahrscheinlich keinen Stau mehr, weil du einsteigst, dein Ziel eingibst, dann rechnet das das wahrscheinlich gut raus, es würde wahrscheinlich sehr viele Probleme lösen, [...] die Autos sparen super viel Sprit, scheiden weniger Abgase aus. Wahrscheinlich ist alles viel flüssiger, du bist viel schneller unterwegs. (B04, Z. 73)
\end{quote}
Äußerungen in diesem Zusammenhang sprachen Fahrassistenzsystemen bzw. autonomen Fahrzeugen zu, dass sie
\begin{itemize}
    \item Besser als Menschen fahren (B01, Z. 96; B04, Z. 69)
    \item Zuverlässigere Teilnehmer im Straßenverkehr sind (B04, Z. 73)
    \item Stau reduzieren (B04, Z. 73)
    \item Besser einparken als Menschen (B05, Z. 31)
\end{itemize}

\subsubsection*{Sicherheit der Technologie}
Die Befragten waren sich nicht einig darüber, inwiefern die Technologie hinter autonomem Fahren bzw. Fahrassistenzsystemen sicher ist. Auf der einen Seite wurden die Systeme als \glqq zum größten Teil sicher\grqq{} (B01, Z. 62) bezeichnet. Auf der anderen Seite wurden von vier der sechs Kandidaten Äußerungen getroffen, die die allgemeine Sicherheit der Systeme in Frage stellen. (B01, Z. 44; B02, Z. 34; B03, Z. 50, 72, 98; B06 Z. 18, 28)
\begin{quote}
    Obwohl ich da glaube ich noch etwas misstrauisch bin, ob das immer so gut funktioniert. (B02, Z. 34)
\end{quote}

Darüber hinaus äußerte ein Befragter starke Bedenken, ob absolute Sicherheit überhaupt gewährleistet werden kann:
\begin{quote}
        Vollendete - also völlige Sicherheit gibt es nur, wenn du dich im Bunker einschließt. (B03, Z. 56)
\end{quote}

\subsubsection*{Vertrauen in Fahrassistenzsysteme und autonome Fahrzeuge}
Zwei der Befragten äußerten sich zu ihrer grundsätzlichen Einstellung gegenüber Fahrassistenzsystemen. Beide haben Vertrauen in Fahrassistenzsysteme, schränken diese Aussage aber auch ein. (B01, Z. 54; B04, Z. 63)

\begin{quote}
    Obwohl ich den Systemen schon sehr vertraue. Also ist jetzt nicht so, dass ich da komplett gegen bin oder so. (B01, Z. 54)
\end{quote}

Die übrigen Befragten haben keine Aussagen getroffen, die eine eindeutige Aussage zu grundsätzlichem Miss- oder Vertrauen zulassen.

\subsubsection*{Hohe Anschaffungskosten der Technologie}
Eine weitere Vorannahme, die bei zwei Kandidaten festgestellt werden konnte ist die Annahme, dass eine neue Technologie wie Fahrassistenzsysteme oder autonome Fahrzeuge in der Anschaffung besonders teuer sind. (B02, Z. 48; B03, Z. 102)

\begin{quote}
    Da die Autos dann wahrscheinlich genauso viel, wenn nicht sogar noch mehr wie ein jetziges, normales Auto kosten sind die für mich dann nicht rentabel. (B03, Z. 102)
\end{quote}

\subsection{Anforderugnen an Fahrassistenzsysteme und autonome Fahrzeuge}
In den Gesprächen nannten die Befragten mehrere Gruppen von Anforderungen, die Fahrassistenzsysteme und autonome Fahrzeuge erfüllen sollen. Diese werden im Folgenden erörtert.

\subsubsection*{Überlegenheit gegenüber dem Menschen}
Zwei der Befragten nannten als Anforderung, dass Fahrassistenzsysteme und insbesondere autonome Fahrzeuge normalen menschlichen Autofahrern in Punkto Fahren mindestens gleichwertig, besser aber überlegen sein müssen. (B03, Z. 94; B04, Z. 105)
\begin{quote}
    sie müssen halt mindestens genausogut, eher sogar noch besser als ein jetziger, normaler Autofahrer sein
\end{quote}

Die Befragten schließen hier auch über die reine Fahrsicherheit hinausgehende Aspekte ein. Explizit genannt wird zwar die Reaktionsfähigkeit (B03, Z. 94) und Wahrnehmung des Umfeldes (B04, Z. 105) aber es ist auch die Rede von \glqq der gesamten Fähigkeit, Auto zu fahren im Straßenverkehr\grqq{}. (B03, Z. 94)

\subsubsection*{Zuverlässiges Erkennen von Situationen}
Dass Fahrassistenzsysteme und autonome Fahrzeuge über die Fähigkeit verfügen, Verkehrssituationen und -teilnehmer insbesonbdere auch unter außergewöhnlichen Umständen korrekt zu erfassen, war für einen der Befragten eine konkrete Anforderung. (B02, Z. 70, 114)

\begin{quote}
    Also die müssen flexibel genug sein, alle Problematiken zu erkennen. Also zum Beispiel [...] ein Fahrrad mit [...] zwei Hinterrädern oder zwei Vorderrädern, eine andere Form, wird anders erkannt… [...] das System muss flexibel genug sein das zu erkennen, wenn auch mal ein Gerät ein bisschen anders aussieht als sonst. (B02, Z. 70)
\end{quote}

\subsubsection*{Vermitteln eines Sicherheitsgefühls}
Zwei Befragte nannten als Anforderung, dass Fahrassistenzsysteme und autonome Fahrzeuge dem Fahrer und auch den anderen Verkehrsteilnehmern ein Gefühl der Sicherheit vermitteln sollten. (B03, Z. 78; B04 Z. 39, 111) In diesem Zusammenhang klingt an, dass solche Systeme ihre Intention oft nicht klar machen; ein Befragter beschwert sich: \glqq kein Mensch fährt halt so\grqq{}. (B04, Z. 39)

\subsubsection*{Möglichkeit für menschliche Intervention}
Eine weitere von mehreren Befragten angeführte Anforderung war, dass der Mensch stets noch die Chance hat, manuell das Fahrgeschehen zu beeinflussen. (B02, Z. 82, 106; B06, Z. 36) Insbesondere war den Befragten dies wichtig, für den Fall, dass \glqq das Fahrzeug Dinge macht, die man nicht möchte\grqq{}. (B02, Z. 82)

\begin{quote}
    Was ich mir vorstellen könnte, dass man trotzdem noch eingreifen kann. (B06, Z. 36)
\end{quote}

Allen Befragten ging es hierbei in letzter Konsequenz hauptsächlich um das Vermeiden von Unfällen.

\subsubsection*{Geringe Ausfallquote}
Ein einzelner Befragter nannte als Anforderung, dass Fahrassistenzsysteme und autonome Fahrzeuge eine möglichst gerine Ausfallquote haben sollten.

\begin{quote}
    Oder besser gesagt geringe Fehlerrate, also keine Ausfall - geringe Ausfallquote. (B03, Z.52)
\end{quote}

\subsubsection*{Unabhängige Prüfstellen}
Einer der Befragten nannte die Anforderung, dass Systeme von unabhängigen, spezialisierten Prüfstellen abgenommen werden sollen. Als Vergleich wird der TÜV genannt.
\begin{quote}
    Ich finde, es sollte halt einfach sowas in Deutschland den TÜV, der sich damit dann hoffentlich auch auskennt sowas prüfen und testen [...]. (B04, Z. 107)
\end{quote}

\subsubsection*{Klar beschriebener Funktionsumfang}
Eine weitere genannte Anforderung ist ein klar beschriebener Funktionsumfang. Das System solle \glqq das tun, was draufsteht\grqq{}. (B03, Z. 52)

\subsection{Gefahren und Probleme}
In der Befragung wurde erhoben, welche möglichen Gefahren und Probleme die Kandidaten im Kontext von Fahrassistenzsystemen und autonomen Fahrzeugen wahrnehmen. Im Folgenden sind die genannten Punkte gelistet.

\subsubsection*{Wartung}
Ein Befragter äußerte Bedenken zu Gefahren, die durch mangelnde Wartung von autonomen oder unterstützenden Fahrsystemen ausgehen. Der Befragte äußerte, dass seiner Einschätzung nach viele Menschen ihre Fahrzeuge generell schon stark vernachlässigen würden. Er sieht ein Gefahrenpotenzial, falls entsprechende Systeme nicht ordnungsgemäß instandgehalten werden. Der Kandidat vergleicht das Gefahrenpotenzial mit Risiken, die durch die TÜV-Kontrollen abgedeckt werden und kommt zu der Einschätzung, dass sich das Niveau des Risikos auf der gleichen Stufe befindet. (B04, Z. 105)

\subsubsection*{Rechtliche Bedenken}
Die Hälfte der Befragten äußerte Bedenken, was rechtliche Fragen im Bezug auf Fahrassistenzsysteme und autonome Fahrzeuge angeht. Ganz allgemein formulierte es einer der Befragten: \glqq Viel Rechtliches. Noch. Weil viel da nicht geklärt ist und ich weiß nicht, wie schnell sich das entwickelt. Aber da muss man nachziehen.\grqq{} (B04, Z. 105)

Ein weiterer Aspekt, der genannt wurde ist der Datenschutz:
\begin{quote}
    Dass [...] Dritte wissen, wo ich mich gerade befinde und viele Zusatzinformationen über mein Auto haben, das würde mich stören. (B02, Z.76)
\end{quote}

Auf die Frage nach der Schuld im Falle eines Unfalls waren die Befragten unterschiedlicher Meinung. Als mögliche Schuldige bei einem Unfall nannten die Befragten:
\begin{itemize}
    \item Den Hersteller (B02, Z. 78)
    \item Den Programmierer der Software, falls die Software fehlerhaft war (B03, Z. 64, 100)
    \item Den Ingenieur des Fahrzeugs, falls die Konstruktion des Fahrzeugs fehlerhaft war (B03, Z. 100)
    \item Dem Fahrer, falls er seine Aufsichtsverantwortung bei einem Fahrassistenzsystem missachtet hat (B01, Z. 94; B04, Z. 75)
    \item Nicht dem Fahrer, falls das Fahrzeug autonom fährt (B04, Z. 75)
    \item Niemandem (B03, Z. 64, 100)
\end{itemize}

Die Befragten geben hier kein einheitliches Meinungsbild. Zu einer sehr differenzierten Einschätzung kommt einer der Befragten auf die Frage nach der Unfallschuld bei einem verunfallten autonomen Fahrzeug:
\begin{quote}
    Hm. Ich weiß nicht, wie - also nicht der Fahrer, meiner Meinung nach - aber ich weiß nicht, wie die Abgrenzung dann im Bereich ist von den verschiedenen Herstellern. Weil ich finde, dann kommt es stark darauf an, was versagt hat: War es Software, war es irgend ein Sensor... Warum, wenn ein Sensor? Liegt es daran, dass der kaputt war und der Fahrer sich nicht darum gekümmert hat, dann ist es doch wieder der Fahrer. Das sind halt viel zu viele kleine Faktoren als dass man das pauschalisieren könnte. (B04, Z. 115)
\end{quote}

\subsubsection*{Moralische Fragen}
Ebenfalls angeführt wurde die Problematik der moralischen Fragen, die im Kontext von autonomen Fahrzeugen und Fahrassistenzsystemen aufkommen. Ein Befragter nannte konkret das Beispiel von unvermeidlichen Unfällen, wo das Fahrzeug selbst eine Entscheidung zwischen zwei Übeln treffen muss. (B03, Z. 90)

\begin{quote}
    Wenn ein Unfall unvermeidbar ist, was macht das autonome Auto. Ja das ist die Moral. (B03, Z. 90)
\end{quote}

\subsubsection*{Überhöhtes Vertrauen in Technik}
Beinahe alle Befragten äußerten, dass Fahrer, die sich zu sehr auf ihr Fahrzeug verlassen ein Risiko darstellen. (B01, Z. 50; B02, Z. 66; B03, Z. 50; B04, Z. 57; B06, Z. 36)

\begin{quote}
    Ich glaub es kann gefährlich sein, wenn man sich zu sehr darauf verlässt und bei diesen Leuchten im Außenspiegel, die aufleuchten, wenn man überholt wird oder wenn ein Auto von hinten kommt, das man sich nur noch darauf verlässt und nicht mehr selber guckt. Und es kann immer sein das es mal nicht funktioniert und die Leute sich zu sehr darauf verlassen [...]. (B06, Z. 36)
\end{quote}

Darüber, wie sich das überhöhte Vertrauen konkret auswirken kann, hatten die Befragten unterschiedliche Einschätzungen:
\begin{itemize}
    \item Erhöhte Unfallgefahr (B01, Z. 50; B02, Z. 66; B03, Z. 50)
    \item Einschränkung/Verwirrung des Fahrers (B01, Z. 50)
    \item Erhöhte Wahrscheinlichkeit von Fahrfehlern (B03, Z. 50)
    \item Verleiten zu Unachtsamkeit (B04, Z.57)
\end{itemize}

\subsubsection*{Unachtsamkeit des Fahrers}
Als weitere Gefahr sehen die Befragten, dass Fahrer durch Fahrassistenzsysteme unachtsam im Straßenverkehr werden. Die Unachtsamkeit muss aber nicht notwendigerweise durch ein überhöhtes Vertrauen in die Assistenzsysteme eines Fahrzeugs bedingt sein, sondern kann auch grundsätzliche Unaufmerksamkeit des Fahrers bedeuten. Besonders hervorgehoben werden außergewöhnliche Situationen, in denen der Fahrer eingreifen müsste, es aber aufgrund von Unachtsamkeit nicht tut. (B04, Z. 57; B05, Z. 25)

Weiterhin genannt werden Assistenzsysteme, die aktiv vom Fahrer beachtet werden müssen, wie z.\,B. Toter-Winkel-Warner. Diese benötigen die Aufmerksamkeit des Fahrers, um wirksam zu sein. (B04, Z. 65)

\subsubsection*{Cyberattacken}
Zwei der Befragten sahen in der Tatsache, dass Fahrzeuge mit Fahrassistenzsystemen oder autonomen Fahrfunktionen häufig mit Netzwerk-Technolgien ausgestattet sind ein Gefahrenpotenzial für Cyberattacken. Explizit genannt wird die \glqq Ausnahmegefahr von Anschlägen\grqq{} (B03, Z.62) auf Grund der erhöhten Angreifbarkeit von zentral gesteuerten Systemen. (B01, Z. 66; B03, Z. 62, 92)

Neben der diffusen Gefahr des Hackerangriffs nannte ein Befragter konkrete Gefährdungsszenarien, die von Cyberattacken auf Fahrzeuge ausgehen: (B03, Z. 92)
\begin{itemize}
    \item Absichtliches Verursachen von Unfällen
    \item Entführen von Fahrzeugen aus der Ferne
\end{itemize}

\subsubsection*{Unzuverlässigkeit anderer (menschlicher) Verkehrsteilnehmer}



\clearpage
\section{Diskussion}

\clearpage
\section{Methodenreflexion}

\clearpage
\section{Fazit}

\clearpage
\pagenumbering{Roman}

\clearpage
\appendix
\pagestyle{appendix}

\label{att:Anhang1}
\fakesection{Anhang1}
% \includepdf[pages={1},scale=0.9,pagecommand={\section{Anhang}}]{}

\section{Literaturverzeichnis}
\pagestyle{MRstyle}
\setcounter{biburllcpenalty}{7000}
\setcounter{biburlucpenalty}{8000}
\printbibliography
\pagebreak

\pagestyle{plain}
\pagenumbering{gobble}
%\includepdf[pagecommand={}]{Eidesstattliche_Versicherung.pdf}

\end{document}
