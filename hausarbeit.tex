\documentclass[fontsize=13pt,a4paper]{article}
\usepackage[utf8]{inputenc}
\usepackage[T1]{fontenc}

\usepackage[dvips]{graphicx}
\usepackage{xcolor}

\usepackage{tgadventor}

\usepackage{setspace}
\onehalfspacing

\usepackage[german]{babel}

\usepackage[
breaklinks=true,colorlinks=true,
linkcolor=black,urlcolor=black,citecolor=black,% PRINT
bookmarks=true,bookmarksopenlevel=2]{hyperref}

\usepackage{geometry}
\geometry{total={210mm,297mm},
left=20mm,right=20mm,
bindingoffset=10mm, top=25mm,bottom=25mm}


\begin{document}
\thispagestyle{empty}

{%%%
 \centering
 \Large

 ~\vspace{\fill}

 {\huge
  Hausarbeit zum Seminar Risikokommunikation}

 \vspace{2.5cm}

 {\Large
  Maximilian Röttgen
 }\\
 Matr. Nr. 332048\\
 {\Large
  Martin Schmitz
}\\
Matr. Nr. 320669

 \vspace{3.5cm}

 RWTH-Aachen University\\
 Professur für Textlinguistik und Technikkommunikation\\
 Institut für Sprach- und Kommunikationswissenschaft

 \vspace{3.5cm}

\vspace{\fill}


 %%%
}%%%
\clearpage

\pagenumbering{Roman}
\tableofcontents
\clearpage

\pagenumbering{arabic}
\section{Einleitung}

\section{Literaturgestützte Einführung}

\section{Methodik}
\subsection{Web-Studie}

\section{Ergebnisse}
Im Folgenden werden die Ergebnisse der Studie vorgestellt. Da die Bewertung der Sicherheit vor den jeweilig persönlichen Hintergründen der Befragten zu betrachten sind, werden zunächst einige Vorinformationen über die Befragten betrachtet.
\subsubsection*{Personendaten}
Unter den sechs Befragten waren fünf Männer und eine Frau. Alle Befragten waren zwischen 23 und 29 Jahren alt, mit einem mittleren Alter von 25 Jahren, Median von 24 Jahren. Die Befragten hatten ihren Führerschein seit mindestens fünf Jahren. Der erfahrenste Fahrer war bereits 12 Jahre im Besitz der Fahrerlaubnis.

Die Hälfte der Befragten war zum Zeitpunkt der Erhebung im Besitz eines eigenen PKW, die anderen 38 \% nutzten geteilte Fahrzeuge, z.\,B. der Eltern oder Car-Sharing Angebote. 


\section{Diskussion}

\section{Methodenreflexion}

\section{Fazit}






\end{document}
