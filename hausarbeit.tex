\documentclass[12pt]{article}
\usepackage[utf8]{inputenc}
\usepackage[T1]{fontenc}
\usepackage[german]{babel}


\usepackage{graphicx}

%   Abbildungen müssen in einem Ordner "Abbildungen" liegen
\graphicspath{ {Abbildungen/} }
\usepackage{pdfpages}
\usepackage{csquotes}

\usepackage{etoolbox}
\AtBeginEnvironment{quote}{\small\setstretch{.25}}

\usepackage{helvet}
\renewcommand{\familydefault}{\sfdefault}
\renewcommand\labelenumi{(\theenumi)}

\usepackage[style=authoryear-icomp,maxcitenames=2,backend=biber]{biblatex}
\addbibresource{bibliography.bib}

\usepackage{titlesec}
\titlespacing\section{0pt}{12pt plus 4pt minus 2pt}{0pt plus 2pt minus 2pt}
\titlespacing\subsection{0pt}{12pt plus 4pt minus 2pt}{0pt plus 2pt minus 2pt}
\titlespacing\subsubsection{0pt}{12pt plus 4pt minus 2pt}{0pt plus 2pt minus 2pt}

\usepackage{setspace}
\usepackage{color}
\definecolor{hgray}{gray}{0.5}

\usepackage{enumitem}
\setlist[itemize]{noitemsep, topsep=0pt}
\setlist[enumerate]{noitemsep, topsep=0pt}

\usepackage[hidelinks,
pdfpagelabels,
pdfstartview = FitH,
bookmarksopen = true,
bookmarksnumbered = true,
linkcolor = black,
plainpages = false,
hypertexnames = false,
citecolor = black] {hyperref}


\usepackage{geometry}
\geometry{
  left=2.5cm,
  right=2.5cm,
  top=2.5cm,
  bottom=2cm,
  bindingoffset=0mm
}


\makeatletter
\title{Aspekte der Wahrnehmung von Sicherheit bei der Fahrer-Fahrzeug-Interaktion}\let\Title\@title     %   Titel der Arbeit eintragen
\makeatother



\usepackage{fancyhdr}
\usepackage{afterpage}
\fancypagestyle{MRstyle}{
    \fancyhf{}
    \fancyhead[L]{\textit{\textcolor{hgray}{\Title}}}
    \fancyfoot[c]{\textcolor{hgray}{\thepage}}
}

\fancypagestyle{appendix}{
    \fancyhf{}
    \fancyhead[L]{\textit{\textcolor{hgray}{\Title}}}
    \fancyhead[R]{\textcolor{hgray}{\rightmark}}
    \fancyfoot[c]{\textcolor{hgray}{\thepage}}
}

\usepackage[hang]{footmisc}

\newcommand\fakesection[1]{%
    \markboth{#1}{#1}}

\usepackage{parskip}
\DefineBibliographyStrings{german}{
   andothers = {{et\,al\adddot}},
}


% Das Dokument geht hier los:

\begin{document}
\pagestyle{MRstyle}

\setstretch{1.15}

\begin{titlepage}

    \large
    RWTH Aachen\\
    Institut für Sprach- und Kommunikationswissenschaft\\
    Professur für Textlinguistik und Technikkommunikation\\
    Prof. Dr. E.-M. Jakobs

    \vspace{5cm}
    \Large
    \doublespacing{
        \textit{Hausarbeit zum Seminar Risikokommunikation\\}
        \textbf{\Title}
    }

    \vspace{7cm}
    \normalsize
    \setstretch{1.2}
    vorgelegt von:\\
    Maximilian Röttgen (Mat.-Nr.: 332048)\\ % Hier Name, Matr. Nr. etc. einfügen
    Martin Schmitz (Mat.-Nr.: 320669)\\
    Joshua Olbrich (Mat.-Nr.: ??????)

    \vfill

    Aachen, \today
\afterpage{\cfoot{\textcolor{hgray}{\thepage}}}

\end{titlepage}


\pagenumbering{Roman}

\tableofcontents
\clearpage
\pagenumbering{gobble}
\section*{Zusammenfassung}

\clearpage
\pagenumbering{arabic}

\section{Einleitung}

\clearpage
\section{Literaturgestützte Einführung}

\clearpage
\section{Methodik}

\subsection{Web-Studie}

\clearpage
\section{Ergebnisse}
Im Folgenden werden die Ergebnisse der Studie vorgestellt. Da die Bewertung der Sicherheit vor den jeweilig persönlichen Hintergründen der Befragten zu betrachten sind, werden zunächst einige Vorinformationen über die Befragten betrachtet.
\subsection*{Personendaten}
Unter den sechs Befragten waren fünf Männer und eine Frau. Alle Befragten waren zwischen 23 und 29 Jahren alt, mit einem mittleren Alter von 25 Jahren, Median von 24 Jahren. Die Befragten hatten ihren Führerschein seit mindestens fünf Jahren. Der erfahrenste Fahrer war bereits 12 Jahre im Besitz der Fahrerlaubnis.

Die Hälfte der Befragten war zum Zeitpunkt der Erhebung im Besitz eines eigenen PKW, die anderen 38\,\% nutzten geteilte Fahrzeuge, z.\,B. der Eltern oder Car-Sharing Angebote.

Familienstand und Einkommen wurden nicht erhoben, die Mehrzahl der Befragten ist jedoch als Student tätig.

\subsection{Kenntnisstand der Befragten}
Der Wissensstand der Befragten zum Thema Fahrassistenzsysteme und autonomes Fahren war zum Zeitpunkt der Befragung gemischt. Auch, wenn die Mehrheit der Befragten selbst noch keine praktische Erfahrung mit entsprechenden Systemen gemacht hat, waren die Befragten durch die Presse (B03, Z. 98; B04, Z. 113) und durch allgemeines Vorwissen (B01, Z. 74ff) zumindest grundsätzlich über das Thema informiert.

Detail- und Anwendungswissen war bei der Hälfte der Befragten nicht vorhanden (B02, Z.44; B03, Z. 30, B05, Z. 24), die restlichen Befragten können  zu einzelnen Assistenzsystemen aus eigenen Erfahrungen berichten (B01, Z. 34, 38; B04 Z. 32-37; B06, Z. 12).

Konkrete Kenntnisse haben die Kandidaten zu folgenden Assistenzsystemen:
\begin{itemize}
    \item Bremsassistent (B01, Z. 34; B04, Z. 29)
    \item Verkehrszeichenerkennung (B01, Z. 38)
    \item Spurhalteassistent (B01, Z. 38; B04, Z. 29)
    \item Abstandshalter (B04, Z. 29)
    \item Tempomat (B04, Z. 29)
    \item Einparkhilfen (B03, Z. 32; B06, Z. 12)
\end{itemize}

Kandidaten, die keine Erfahrungen mit Fahrassistenzsystemen oder autonomen Fahrzeugen machen konnten, nannten dafür folgende Gründe:
\begin{itemize}
    \item Kein Zugang zu entsprechenden Fahrzeugen (B03, Z. 32, 78; B06, Z. 12)
    \item Eigener PKW nicht mit Fahrassistenzsystemen ausgestattet (B02, Z. 46)
    \item Fahrassistenzsysteme kein Kaufkriterium (B06, Z. 18)
\end{itemize}

\subsection{Vorannahmen der Befragten}
Einige Äußerungen der Kandidaten ließen auf Vorannahmen bezüglich autonomem und unterstütztem Fahren schließen. Im Folgenden sind diese Vorannahmen aufgelistet und erläutert. Diese Vorannahmen dienen als Kontext zur besseren Einordnung der Ergebnisse.
\subsubsection*{Berichterstattung in der Presse}
Ein Befragter äußerte, dass er von einer negativen Färbung der Berichterstattung in den Medien über das Thema autonomes/assistiertes Fahren ausgeht.
\begin{quote}
    Meistens kommt ja auch nur das Negative dann in die Presse und nicht das Auto fährt jetzt seit 100.000 Km autonom erfolgreich. Nein, da kommt was, wenn es einen übergebretzelt hat. (B04, Z. 113)
\end{quote}

\subsubsection*{Überlegenheit von Technik}
Es gab weiterhin Äußerungen der Befragten, die nahelegen, dass Grundannahmen darüber vorherrschen, ob und wenn ja in welchen Fällen die Technologie besser ist als der Mensch.
\begin{quote}
    Wenn [...] wirklich alle Autos miteinander vernetzt wären, dann hättest du wahrscheinlich keinen Stau mehr, weil du einsteigst, dein Ziel eingibst, dann rechnet das das wahrscheinlich gut raus, es würde wahrscheinlich sehr viele Probleme lösen, [...] die Autos sparen super viel Sprit, scheiden weniger Abgase aus. Wahrscheinlich ist alles viel flüssiger, du bist viel schneller unterwegs. (B04, Z. 73)
\end{quote}
Äußerungen in diesem Zusammenhang sprachen Fahrassistenzsystemen bzw. autonomen Fahrzeugen zu, dass sie
\begin{itemize}
    \item Besser als Menschen fahren (B01, Z. 96; B04, Z. 69)
    \item Zuverlässigere Teilnehmer im Straßenverkehr sind (B04, Z. 73)
    \item Stau reduzieren (B04, Z. 73)
    \item Besser einparken als Menschen (B05, Z. 31)
\end{itemize}

\subsubsection*{Sicherheit der Technologie}
Die Befragten waren sich nicht einig darüber, inwiefern die Technologie hinter autonomem Fahren bzw. Fahrassistenzsystemen sicher ist. Auf der einen Seite wurden die Systeme als \glqq zum größten Teil sicher\grqq{} (B01, Z. 62) bezeichnet. Auf der anderen Seite wurden von vier der sechs Kandidaten Äußerungen getroffen, die die allgemeine Sicherheit der Systeme in Frage stellen. (B01, Z. 44; B02, Z. 34; B03, Z. 50, 72, 98; B06 Z. 18, 28)
\begin{quote}
    Obwohl ich da glaube ich noch etwas misstrauisch bin, ob das immer so gut funktioniert. (B02, Z. 34)
\end{quote}

Darüber hinaus äußerte ein Befragter starke Bedenken, ob absolute Sicherheit überhaupt gewährleistet werden kann:
\begin{quote}
        Vollendete - also völlige Sicherheit gibt es nur, wenn du dich im Bunker einschließt. (B03, Z. 56)
\end{quote}

\subsubsection*{Vertrauen in Fahrassistenzsysteme und autonome Fahrzeuge}
Zwei der Befragten äußerten sich zu ihrer grundsätzlichen Einstellung gegenüber Fahrassistenzsystemen. Beide haben Vertrauen in Fahrassistenzsysteme, schränken diese Aussage aber auch ein. (B01, Z. 54; B04, Z. 63)

\begin{quote}
    Obwohl ich den Systemen schon sehr vertraue. Also ist jetzt nicht so, dass ich da komplett gegen bin oder so. (B01, Z. 54)
\end{quote}

Die übrigen Befragten haben keine Aussagen getroffen, die eine eindeutige Aussage zu grundsätzlichem Miss- oder Vertrauen zulassen.

\subsubsection*{Hohe Anschaffungskosten der Technologie}
Eine weitere Vorannahme, die bei zwei Kandidaten festgestellt werden konnte ist die Annahme, dass eine neue Technologie wie Fahrassistenzsysteme oder autonome Fahrzeuge in der Anschaffung besonders teuer sind. (B02, Z. 48; B03, Z. 102)

\begin{quote}
    Da die Autos dann wahrscheinlich genauso viel, wenn nicht sogar noch mehr wie ein jetziges, normales Auto kosten sind die für mich dann nicht rentabel. (B03, Z. 102)
\end{quote}

\subsection{Anforderugnen an Fahrassistenzsysteme und autonome Fahrzeuge}
In den Gesprächen nannten die Befragten mehrere Gruppen von Anforderungen, die Fahrassistenzsysteme und autonome Fahrzeuge erfüllen sollen. Diese werden im Folgenden erörtert.

\subsubsection*{Überlegenheit gegenüber dem Menschen}
Zwei der Befragten nannten als Anforderung, dass Fahrassistenzsysteme und insbesondere autonome Fahrzeuge normalen menschlichen Autofahrern in Punkto Fahren mindestens gleichwertig, besser aber überlegen sein müssen. (B03, Z. 94; B04, Z. 105)
\begin{quote}
    sie müssen halt mindestens genausogut, eher sogar noch besser als ein jetziger, normaler Autofahrer sein
\end{quote}

Die Befragten schließen hier auch über die reine Fahrsicherheit hinausgehende Aspekte ein. Explizit genannt wird zwar die Reaktionsfähigkeit (B03, Z. 94) und Wahrnehmung des Umfeldes (B04, Z. 105) aber es ist auch die Rede von \glqq der gesamten Fähigkeit, Auto zu fahren im Straßenverkehr\grqq{}. (B03, Z. 94)

\subsubsection*{Zuverlässiges Erkennen von Situationen}
Dass Fahrassistenzsysteme und autonome Fahrzeuge über die Fähigkeit verfügen, Verkehrssituationen und -teilnehmer insbesonbdere auch unter außergewöhnlichen Umständen korrekt zu erfassen, war für einen der Befragten eine konkrete Anforderung. (B02, Z. 70, 114)

\begin{quote}
    Also die müssen flexibel genug sein, alle Problematiken zu erkennen. Also zum Beispiel [...] ein Fahrrad mit [...] zwei Hinterrädern oder zwei Vorderrädern, eine andere Form, wird anders erkannt… [...] das System muss flexibel genug sein das zu erkennen, wenn auch mal ein Gerät ein bisschen anders aussieht als sonst. (B02, Z. 70)
\end{quote}

\subsubsection*{Vermitteln eines Sicherheitsgefühls}
Zwei Befragte nannten als Anforderung, dass Fahrassistenzsysteme und autonome Fahrzeuge dem Fahrer und auch den anderen Verkehrsteilnehmern ein Gefühl der Sicherheit vermitteln sollten. (B03, Z. 78; B04 Z. 39, 111) In diesem Zusammenhang klingt an, dass solche Systeme ihre Intention oft nicht klar machen; ein Befragter beschwert sich: \glqq kein Mensch fährt halt so\grqq{}. (B04, Z. 39)

\subsubsection*{Unabhängige Prüfstellen}
Einer der Befragten nannte die Anforderung, dass Systeme von unabhängigen, spezialisierten Prüfstellen abgenommen werden sollen. Als Vergleich wird der TÜV genannt.
\begin{quote}
    Ich finde, es sollte halt einfach sowas in Deutschland den TÜV, der sich damit dann hoffentlich auch auskennt sowas prüfen und testen [...]. (B04, Z. 107)
\end{quote}



\clearpage
\section{Diskussion}

\clearpage
\section{Methodenreflexion}

\clearpage
\section{Fazit}

\clearpage
\pagenumbering{Roman}

\clearpage
\appendix
\pagestyle{appendix}

\label{att:Anhang1}
\fakesection{Anhang1}
% \includepdf[pages={1},scale=0.9,pagecommand={\section{Anhang}}]{}

\section{Literaturverzeichnis}
\pagestyle{MRstyle}
\setcounter{biburllcpenalty}{7000}
\setcounter{biburlucpenalty}{8000}
\printbibliography
\pagebreak

\pagestyle{plain}
\pagenumbering{gobble}
%\includepdf[pagecommand={}]{Eidesstattliche_Versicherung.pdf}

\end{document}
