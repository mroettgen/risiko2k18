\documentclass[12pt]{article}
\usepackage[utf8]{inputenc}
\usepackage[T1]{fontenc}
\usepackage[german]{babel}


\usepackage{graphicx}

%   Abbildungen müssen in einem Ordner "Abbildungen" liegen
\graphicspath{ {Abbildungen/} }
\usepackage{pdfpages}
\usepackage{csquotes}

\usepackage{etoolbox}
\AtBeginEnvironment{quote}{\small\setstretch{.25}}

\usepackage{helvet}
\renewcommand{\familydefault}{\sfdefault}
\renewcommand\labelenumi{(\theenumi)}

\usepackage[style=apa, maxcitenames=2, citestyle=authoryear, backend=biber]{biblatex}
\addbibresource{bibliography.bib}

\DeclareNameAlias{sortname}{last-first}
\DeclareNameAlias{default}{last-first}

\usepackage{titlesec}
\titlespacing\section{0pt}{12pt plus 4pt minus 2pt}{0pt plus 2pt minus 2pt}
\titlespacing\subsection{0pt}{12pt plus 4pt minus 2pt}{0pt plus 2pt minus 2pt}
\titlespacing\subsubsection{0pt}{12pt plus 4pt minus 2pt}{0pt plus 2pt minus 2pt}

\usepackage{booktabs}

\usepackage{setspace}
\usepackage{color}
\definecolor{hgray}{gray}{0.5}

\usepackage{enumitem}
\setlist[itemize]{noitemsep, topsep=0pt}
\setlist[enumerate]{noitemsep, topsep=0pt}

\usepackage[hidelinks,
  pdfpagelabels,
  pdfstartview = FitH,
  bookmarksopen = true,
  bookmarksnumbered = true,
  linkcolor = black,
  plainpages = false,
  hypertexnames = false,
  citecolor = black] {hyperref}


\usepackage{geometry}
\geometry{
  left=2.5cm,
  right=3cm,
  top=2.5cm,
  bottom=2.5cm,
  bindingoffset=0mm
}


\makeatletter
\title{Aspekte der Wahrnehmung von Sicherheit bei der Fahrer-Fahrzeug-Interaktion}\let\Title\@title     %   Titel der Arbeit eintragen
\makeatother



\usepackage{fancyhdr}
\usepackage{afterpage}
\fancypagestyle{MRstyle}{
  \fancyhf{}
  \fancyhead[L]{\textit{\textcolor{hgray}{\Title}}}
  \fancyfoot[c]{\textcolor{hgray}{\thepage}}
}

\fancypagestyle{appendix}{
  \fancyhf{}
  \fancyhead[L]{\textit{\textcolor{hgray}{\Title}}}
  \fancyhead[R]{\textcolor{hgray}{\rightmark}}
  \fancyfoot[c]{\textcolor{hgray}{\thepage}}
}

\usepackage[hang]{footmisc}

\newcommand\fakesection[1]{%
  \markboth{#1}{#1}}

\usepackage{parskip}
\DefineBibliographyStrings{german}{
  andothers = {{et\,al\adddot}},
}


% Das Dokument geht hier los:

\begin{document}
\pagestyle{MRstyle}

\setstretch{1.5}

\begin{titlepage}

  \large
  RWTH Aachen\\
  Institut für Sprach- und Kommunikationswissenschaft\\
  Professur für Textlinguistik und Technikkommunikation\\
  Prof. Dr. E.-M. Jakobs

  \vspace{5cm}
  \Large
  \doublespacing{
    \textit{Hausarbeit zum Seminar Risikokommunikation\\}
    \textbf{\Title}
  }

  \vspace{7cm}
  \normalsize
  \setstretch{1.2}
  vorgelegt von:\\
  Maximilian Röttgen (Mat.-Nr.: 332048)\\ % Hier Name, Matr. Nr. etc. einfügen
  Martin Schmitz (Mat.-Nr.: 320669)\\
  Joshua Olbrich (Mat.-Nr.: 331461)

  \vfill

  Aachen, \today
  \afterpage{\cfoot{\textcolor{hgray}{\thepage}}}

\end{titlepage}


\pagenumbering{Roman}

\tableofcontents
\clearpage
\pagenumbering{gobble}
\section*{Zusammenfassung}

\clearpage
\pagenumbering{arabic}

\section{Einleitung}

\clearpage
\section{Literaturgestützte Einführung}

\subsection{Risiko und Sicherheit}
Sicherheit ist ein Begriff, für den beinahe jedes Berufsfeld eine eigene Nuancierung hat. Dadurch unterscheidet sich die Definition von Sicherheit zwischen verschiedenen Disziplinen und Kontexten. Im folgenden Abschnitt soll ein kurzer Überblick über unterschiedliche Definitionen von Sicherheit gegeben werden.

%\subsubsection*{Sicherheit und Akzeptanz}

Rothkegel definiert vier verschiedene Sicherheits-Modelle, die bei der Sicherheitskommunikation genutzt werden können, um \glqq \glq gelingende\grq \, bzw. \glq gedeihliche\grq \, Kommunikation zum Thema Sicherheit\grqq \, herzustellen (\cite[125]{rothkegel2013sicherheitskommunikation}):
\begin{itemize}
  \item Sicherheit als Abwesenheit von Risiko und Gefahr
  \item Sicherheit als Umgang mit und Steuerung von Risiken
  \item Sicherheit als Umgang mit und Steuerung von Gefahren und
  \item Sicherheit als (interner) Selbstschutz
\end{itemize}

Bei der Definition von Sicherheit als Umgang mit und Steuerung von Risiken wird oft eine Einordnung des Risikos durch Berechnung der Auftretenswahrscheinlichkeit des Schadensausmaßes vorgenommen (vgl. ebd.). Aufgrund dieses Kalküls wird anschließend entschieden, ob man beispielsweise als Hersteller oder Verbraucher gewillt ist, das Risiko einzugehen.

Bei der Definition von Sicherheit als Umgang mit und Steuerung von Gefahren geht es darum, die Gefahr zu minimieren und einzudämmen. Hier wird zwischen aktiver und passiver Sicherheit unterscheiden. Aktive Sicherheit wehrt die Gefahr ab -- etwa durch ABS-Systeme in einem Auto, die das Bremsen sicherer gestalten. Passive Sicherheit schützt dagegen vor dem Schadenseintritt. Ein Beispiel wären Airbags, die einen Autounfall nicht verhindern, den Fahrer im Falle eines solchen Unfalls aber schützen (vgl. ebd., S. 130).

Rothkegel kommt in ihrer Arbeit zu dem Schluss, dass in von Herstellern ausgelösten Kommunikationssituationen \glqq kein Raum für das Thema Sicherheit\grqq \, sei (\cite[132]{rothkegel2013sicherheitskommunikation}). Stattdessen fokussiert eine solche Kommunikation auf das Erlebnis sowie auf Wunschdenken. Wird doch in der Öffentlichkeit über Risiken gesprochen, kommt es zu einer Mischung von Fach- und Alltagswissen und von Fach- und Alltagssprache, was in Kommunikationsproblemen resultiert (vgl. \cite[134]{rothkegel2013sicherheitskommunikation}). Rothkegel schließt daraus, dass \glqq [e]ine Kommunikationskultur, in der der Begriff des Risikos die Sicherheitskommunikation bestimmt, [...] per se konflikthaft [ist]\grqq \, (ebd., S. 135).

Laut Banse nimmt die allgemeine Risikoakzeptanz gesellschaftsweit ab, während das Sicherheitsverlangen im gleichen Maße zunimmt (vgl. \cite[3]{banse2018technik}). Dieses Verlangen nach Sicherheit kann dabei nicht nur Statistiken, Zahlen, Daten und Fakten befriedigt werden. Das liegt daran, dass Sicherheit nicht lediglich aus rationalem Wissen entstehe, sondern auch aus \glqq einem intuitiven Verständnis, aus Erfahrungen und Erwartungen, aus Hoffnungen und Ängsten, aus erlebten Mitgestaltungsmöglichkeiten bei technischen Problemlösungsprozessen oder zumindest wahrgenommenen Eingriffsmöglichkeiten in technische Abläufe bzw. aus Ohnmachtsgefühlen angesichts einer scheinbaren Eigendynamik des Technischen\grqq \, (\cite[4]{banse2018technik}).

Genau eine solche Eigendynamik scheinen dabei immer mehr Systeme zu entwickeln. Viele technische Geräte scheinen selbst für versierte Nutzer unberechenbar zu sein (vgl. \cite[5]{norman2013design}), was zu Frustration und Skepsis gegenüber solche Systeme führt. Dazu kommt, dass bei vielen komplexen technologischen Lösungen eine Begrenzung der Folgen fast nicht durchzuführen ist und auch genaues Wissen über Schadensausmaß und Eintrittswahrscheinlichkeit kaum zu ermitteln ist (vgl. \cite[12]{banse2018technik}).

Einen Wandel des Risikobegriffs in den vergangenen Jahrhunderten bemerkt Lau (1989). Er stellt fest, dass sich das gestiegene Gefahrenbewusstsein vor allem gegen Konsequenzen technologischer Entwicklungen wendet (vgl. \cite[418]{lau1989risikodiskurse}). Einen Grund dafür sieht er darin, dass sich der Begriff des Risikos geändert hat. Früher waren Risiken \glqq Experimente mit der eigenen Person\grqq \, waren (vgl. ebd., S. 421), etwa wenn ein Seemann nach Indien aufbrach und damit bewusst ein Risiko einging. Epidemien, Unfälle, Kriege und Ähnliches wurden dagegen als \glqq allgemeine irdische Lebensgefährdungen\grqq \, aufgefasst (vgl. ebd.).

Dies änderte sich mit dem Aufkommen des Versicherungswesens. Risiken wurden berechnet und quantifiziert. Sie waren nicht länger ein \glq Experiment\grq , das im Falle eines Erfolgs Ruhm, Reichtum oder Erfahrung mit sich brachte. Vielmehr handelte es um Ereignisse wie etwa das Abbrennen des eigenen Wohnhauses, das von der Versicherung gedeckt wurde (vgl. ebd., S. 422).

Heutige, durch Technologie ausgelöste Risiken stellen laut Lau eine Mischform dieser beiden Risikoverständnisse dar. Sie werden zwar nicht freiwillig eingegangen, haben ihre Ursachen aber im Entscheiden und Handeln von Personen oder Institutionen (vgl. ebd., S. 423). Damit sind solche Risiken gleichzeitig auf menschliches Handeln zurückzuführen und haben für Betroffene die gleiche Anonymität wie etwa Naturkatastrophen, wodurch sie laut Lau eine \glqq soziale Sprengkraft\grqq \, entfalten (vgl. ebd., S. 433). Eine Möglichkeit, solchen Risiken diese soziale Sprengkraft zu nehmen ist es, solche technologisch erzeugten Risiken zu \glq natürlichen Gefahren\grq \, umzudefinieren (vgl. ebd.), also zu zeigen, dass das durch die Technologie ausgelöste Risiko vergleichbar (oder sogar besser) ist als ein ähnliches, natürliches Risiko.

\subsection{Risikokommunikation}
Kasperson (2014) fasst zusammen, dass in den letzten 30 Jahren zwar viel theoretische und wissenschaftliche Arbeit auf dem Feld der Risikokommunikation geleistet wurde, sich in der Praxis bislang aber kaum etwas geändert habe (vgl. \cite[1234]{kasperson2014four}). Er stellt vier Prinzipien auf, die die Risikokommunikation erfolgreicher gestalten sollen (vgl. ebd., S. 1237f.):
\begin{enumerate}
  \item Der Risikokommunikation eines Projektes muss wegen der gestiegenen Anforderungen in der Bevölkerung mehr Mittel zur Verfügung gestellt bekommen und ambitioniertere Ziele verfolgen als dies bisher häufig der Fall ist
  \item Die Kommunikation selbst soll nicht nur auf Expertenebene erfolgen, sondern Konsumenten auch in ihrem täglichen Leben erreichen
  \item Je größer die Ungewissheiten betreffend eines Risikos sind, desto mehr muss kommuniziert werden. Außerdem muss kommuniziert werden, welche Risiken sich in welchem Zeitraum senken lassen können
  \item Ziele, Struktur und Durchführung der Risikokommunikation müssen dem bestehenden sozialen Mistrauen angepasst werden. Dabei muss darauf eingegangen werden, dass innerhalb der Bevölkerung das Vertrauen in Institutionen in den letzten Jahren erheblich geschrumpft ist (vgl. ebd., S. 1236)
\end{enumerate}

Sehr ähnliche Anforderungen stellt auch Renn (2010) auf. Er definiert als Aufgabe der Risikokommunikation, Menschen mit genügend Wissen auszustatten, damit sie selbstständig fundierte Entscheidungen treffen können (vgl. \cite[81]{renn2010risk}). Da laut ihm die Anforderungen an Risikokommunikateure in den vergangen Jahren gewachsen seien, müssten Unternehmen und Regierungen der Öffentlichkeit heute mehr Informationen zur Verfügung stellen als früher (vgl. ebd., S. 82ff.). Dabei gibt es drei Ebenen, die bei solchen \glq Risiko-Debatten\grq \, beachtet werden müssen:
\begin{itemize}
  \item \textbf{Fakten und Wahrscheinlichkeiten:} Risikokommunikateure müssen der Öffentlichkeit nicht nur Zahlen, Daten und Fakten präsentieren, sondern auch dabei unterstützen diese selbstständig interpretieren zu können
  \item \textbf{Expertise, Erfahrung und Leistung des Unternehmens:} Hierfür muss ein Dialog zwischen den Stakeholdern und der Öffentlichkeit hergestellt werden
  \item \textbf{Konflikte mit bestehenden, persönlichen Wertesystemen und Erfahrungen:} Diese Ebene ist laut Renn die am schwierigsten zu erreichende. Der Umgang mit persönlichen Wertevorstellungen und Erfahrungen erfordert ein hohes Maß an Empathie und erfordert eine erhöhte Nahbarkeit der Stakeholder
\end{itemize}

Insgesamt fasst Renn zusammen, dass Risikokommunikation über bloße PR-Arbeit und Informationen hinausgeht und die Bürger auch auf einer persönlichen Ebene erreichen muss (vgl. ebd., S. 95).

\subsection{Autonomes Fahren}

Autonome Fahrzeuge werden in den kommenden Jahren aller Voraussicht nach ein integraler Bestandteil des Straßenverkehrs werden. Neben gesteigertem Fahrkomfort können solche selbstfahrenden Autos auch die Sicherheit im Straßenverkehr positiv beeinflussen – unter anderem, indem Unfälle durch menschliches Versagen verhindert werden und ältere und behinderte Menschen in ihrer Mobilität unterstützt werden können (vgl.  \cite[167]{fagnant2015preparing}).

Tatsächlich zeigte eine Studie der National Highway Traffic Safety Administration, dass 90\% der Unfälle, die in der Unfallstatistik gelistet wurden, durch menschliches Versagen verursacht wurden. 40\% der Unfälle mit tödlichem Ausgang waren auf Alkohol- oder Drogenkonsum, Ablenkung oder Müdigkeit zurückzuführen (vgl. \cite{singh2015critical}). Hier könnten selbstfahrende Fahrzeuge für eine erhöhte Sicherheit für alle Straßenverkehrsteilnehmer sorgen: \glqq Self-driven Self-driven vehicles would not fall prey to human failings, suggesting the potential for at least a 40\% fatal crash-rate reduction, assuming automated malfunctions are minimal and everything else remains constant. Such reductions do not reflect crashes due to speeding, aggressive driving, over-compensation, inexperience, slow reaction times, inattention and various other driver shortcomings.\grqq \ (\cite[169]{fagnant2015preparing}).

In der Zukunft werden Autos höchstwahrscheinlich miteinander vernetzt sein und stets miteinander kommunizieren. Dies wird die Notwendigkeit eines menschlichen Fahrers weiter minimieren (vgl. \cite[241]{gerla2014internet}). Mit diesem Schritt weg vom individuellen Vehikel und hin zu einer stets über die \emph{Cloud} autonomen Fahrzeugflotte kann die Sicherheit im Straßenverkehr noch weiter erhöht werden (vgl. ebd.).

Doch obwohl diese neue Technologie rein statistisch betrachtet sicherer zu sein scheint, herrscht viel Skepsis gegenüber dem autonomen Fahren. Dies liegt unter daran, dass es gerade bei neuartigen Technologien oft eine Diskrepanz zwischen tatsächlichem und wahrgenommenem Risiko besteht (vgl. \cite[106]{hengstler2016applied}). Eine Möglichkeit, dieses wahrgenommene Risiko zu verkleinern, ist, Vertrauen in das Produkt zu schaffen (vgl. \cite{rousseau1998not}). Doch nicht nur das Vertrauen in die Technologie an sich ist ein entscheidender Faktor zur allgemeinen Akzeptanz, auch das Vertrauen in den \emph{Hersteller} ist wichtig (vgl. \cite[107]{hengstler2016applied}). Hengstler et al. kommen dabei zu dem Schluss, dass eine gute Kommunikation seitens des Herstellers positive Effekte auf das Vertrauen in den Hersteller und damit auch auf das Vertrauen in die neue Technologie haben kann (ebd.).


\clearpage
\section{Methodik}
Dieser Abschnitt befasst sich mit der gewählten Methode der qualitativen Inhaltsanalyse. Dabei sollen die einzelnen Schritte der Methode näher betrachtet und in die zugrunde liegende eingeordnet eingeordnet werden. Es soll ersichtlich werden, warum die Methode gewählt wurde und die Transparenz zum nachvollziehen der Analyseschritte
Zur Untersuchung der “Wahrnehmung von Sicherheit bei der Fahrer-Fahrzeug-Interaktion” wurde ein qualitativer Forschungsansatz, in Form einer Interviewstudie, gewählt. Mittels einer qualitativen Studie das Forschungsthema “Wahrnehmung von Sicherheit” zu untersuchen, ist für die Thematik ratsam, da Wahrnehmung und Sicherheit zwei sehr subjektive Felder sind. Somit ist es schwer eine urteilsfähige Parametrisierung zu finden. Durch eine Interviewstudie ist es möglich einen offenen und unvoreingenommenen Einblick in die Thematik und deren Einbettung in den Alltag der Befragten, mit aller Komplexität, zu erhalten (Flick 1995 S. 14).
Die gewählte Methode folgt der qualitativen Inhaltsanalyse nach Mayring. Die von ihm entwickelten Methodik soll die geführten Interviews schrittweise, systematisch und theoriegeleitet analysieren. Kernelement seiner Methode ist das bestimmen von Kategorien, in denen im Interview gefallene Aussagen zusammengefasst werden. Oberkategorien werder zuerst anhand von Theorie deduktiv abgeleiteten und später anhand von des gesammelten Materiales induktiv erweitert und verfeinert.



\subsection{Web-Studie}

\clearpage
\section{Ergebnisse}
Im Folgenden werden die Ergebnisse der Studie vorgestellt. Da die Bewertung der Sicherheit vor den jeweilig persönlichen Hintergründen der Befragten zu betrachten sind, werden zunächst einige Vorinformationen über die Befragten betrachtet.
\subsection*{Personendaten}
Unter den sechs Befragten waren fünf Männer und eine Frau. Alle Befragten waren zwischen 23 und 29 Jahren alt, mit einem mittleren Alter von 25 Jahren, Median von 24 Jahren. Die Befragten hatten ihren Führerschein seit mindestens fünf Jahren. Der erfahrenste Fahrer war bereits 12 Jahre im Besitz der Fahrerlaubnis.

Die Hälfte der Befragten war zum Zeitpunkt der Erhebung im Besitz eines eigenen PKW, die anderen 38\,\% nutzten geteilte Fahrzeuge, z.\,B. der Eltern oder Car-Sharing Angebote.

Familienstand und Einkommen wurden nicht erhoben, die Mehrzahl der Befragten ist jedoch als Student tätig.

\subsection{Kenntnisstand der Befragten}
Der Wissensstand der Befragten zum Thema Fahrassistenzsysteme und autonomes Fahren war zum Zeitpunkt der Befragung gemischt. Auch, wenn die Mehrheit der Befragten selbst noch keine praktische Erfahrung mit entsprechenden Systemen gemacht hat, waren die Befragten durch die Presse (B03, Z. 98; B04, Z. 113) und durch allgemeines Vorwissen (B01, Z. 74ff) zumindest grundsätzlich über das Thema informiert.

Detail- und Anwendungswissen war bei der Hälfte der Befragten nicht vorhanden (B02, Z.44; B03, Z. 30, B05, Z. 24), die restlichen Befragten können  zu einzelnen Assistenzsystemen aus eigenen Erfahrungen berichten (B01, Z. 34, 38; B04 Z. 32-37; B06, Z. 12).

Konkrete Kenntnisse haben die Kandidaten zu folgenden Assistenzsystemen:
\begin{itemize}
  \item Bremsassistent (B01, Z. 34; B04, Z. 29)
  \item Verkehrszeichenerkennung (B01, Z. 38)
  \item Spurhalteassistent (B01, Z. 38; B04, Z. 29)
  \item Abstandshalter (B04, Z. 29)
  \item Tempomat (B04, Z. 29)
  \item Einparkhilfen (B03, Z. 32; B06, Z. 12)
\end{itemize}

Kandidaten, die keine Erfahrungen mit Fahrassistenzsystemen oder autonomen Fahrzeugen machen konnten, nannten dafür folgende Gründe:
\begin{itemize}
  \item Kein Zugang zu entsprechenden Fahrzeugen (B03, Z. 32, 78; B06, Z. 12)
  \item Eigener PKW nicht mit Fahrassistenzsystemen ausgestattet (B02, Z. 46)
  \item Fahrassistenzsysteme kein Kaufkriterium (B06, Z. 18)
\end{itemize}


\subsection{Vorannahmen der Befragten}
Einige Äußerungen der Kandidaten ließen auf Vorannahmen bezüglich autonomem und unterstütztem Fahren schließen. Im Folgenden sind diese Vorannahmen aufgelistet und erläutert. Diese Vorannahmen dienen als Kontext zur besseren Einordnung der Ergebnisse.
\subsubsection*{Berichterstattung in der Presse}
Ein Befragter äußerte, dass er von einer negativen Färbung der Berichterstattung in den Medien über das Thema autonomes/assistiertes Fahren ausgeht.
\begin{quote}
  Meistens kommt ja auch nur das Negative dann in die Presse und nicht das Auto fährt jetzt seit 100.000 Km autonom erfolgreich. Nein, da kommt was, wenn es einen übergebretzelt hat. (B04, Z. 113)
\end{quote}

\subsubsection*{Überlegenheit von Technik}
Es gab weiterhin Äußerungen der Befragten, die nahelegen, dass Grundannahmen darüber vorherrschen, ob und wenn ja in welchen Fällen die Technologie besser ist als der Mensch.
\begin{quote}
  Wenn [...] wirklich alle Autos miteinander vernetzt wären, dann hättest du wahrscheinlich keinen Stau mehr, weil du einsteigst, dein Ziel eingibst, dann rechnet das das wahrscheinlich gut raus, es würde wahrscheinlich sehr viele Probleme lösen, [...] die Autos sparen super viel Sprit, scheiden weniger Abgase aus. Wahrscheinlich ist alles viel flüssiger, du bist viel schneller unterwegs. (B04, Z. 73)
\end{quote}
Äußerungen in diesem Zusammenhang sprachen Fahrassistenzsystemen bzw. autonomen Fahrzeugen zu, dass sie
\begin{itemize}
  \item Besser als Menschen fahren (B01, Z. 96; B04, Z. 69)
  \item Zuverlässigere Teilnehmer im Straßenverkehr sind (B04, Z. 73)
  \item Stau reduzieren (B04, Z. 73)
  \item Besser einparken als Menschen (B05, Z. 31)
\end{itemize}

\subsubsection*{Sicherheit der Technologie}
Die Befragten waren sich nicht einig darüber, inwiefern die Technologie hinter autonomem Fahren bzw. Fahrassistenzsystemen sicher ist. Auf der einen Seite wurden die Systeme als \glqq zum größten Teil sicher\grqq{} (B01, Z. 62) bezeichnet. Auf der anderen Seite wurden von vier der sechs Kandidaten Äußerungen getroffen, die die allgemeine Sicherheit der Systeme in Frage stellen. (B01, Z. 44; B02, Z. 34; B03, Z. 50, 72, 98; B06 Z. 18, 28)
\begin{quote}
  Obwohl ich da glaube ich noch etwas misstrauisch bin, ob das immer so gut funktioniert. (B02, Z. 34)
\end{quote}

Darüber hinaus äußerte ein Befragter starke Bedenken, ob absolute Sicherheit überhaupt gewährleistet werden kann:
\begin{quote}
  Vollendete - also völlige Sicherheit gibt es nur, wenn du dich im Bunker einschließt. (B03, Z. 56)
\end{quote}

\subsubsection*{Vertrauen in Fahrassistenzsysteme und autonome Fahrzeuge}
Zwei der Befragten äußerten sich zu ihrer grundsätzlichen Einstellung gegenüber Fahrassistenzsystemen. Beide haben Vertrauen in Fahrassistenzsysteme, schränken diese Aussage aber auch ein. (B01, Z. 54; B04, Z. 63)

\begin{quote}
  Obwohl ich den Systemen schon sehr vertraue. Also ist jetzt nicht so, dass ich da komplett gegen bin oder so. (B01, Z. 54)
\end{quote}

Die übrigen Befragten haben keine Aussagen getroffen, die eine eindeutige Aussage zu grundsätzlichem Miss- oder Vertrauen zulassen.

\subsubsection*{Hohe Anschaffungskosten der Technologie}
Eine weitere Vorannahme, die bei zwei Kandidaten festgestellt werden konnte ist die Annahme, dass eine neue Technologie wie Fahrassistenzsysteme oder autonome Fahrzeuge in der Anschaffung besonders teuer sind. (B02, Z. 48; B03, Z. 102)

\begin{quote}
  Da die Autos dann wahrscheinlich genauso viel, wenn nicht sogar noch mehr wie ein jetziges, normales Auto kosten sind die für mich dann nicht rentabel. (B03, Z. 102)
\end{quote}


\subsection{Anforderugnen an Fahrassistenzsysteme und autonome Fahrzeuge}
In den Gesprächen nannten die Befragten mehrere Gruppen von Anforderungen, die Fahrassistenzsysteme und autonome Fahrzeuge erfüllen sollen. Diese werden im Folgenden erörtert.

\subsubsection*{Überlegenheit gegenüber dem Menschen}
Zwei der Befragten nannten als Anforderung, dass Fahrassistenzsysteme und insbesondere autonome Fahrzeuge normalen menschlichen Autofahrern in Punkto Fahren mindestens gleichwertig, besser aber überlegen sein müssen. (B03, Z. 94; B04, Z. 105)
\begin{quote}
  sie müssen halt mindestens genausogut, eher sogar noch besser als ein jetziger, normaler Autofahrer sein
\end{quote}

Die Befragten schließen hier auch über die reine Fahrsicherheit hinausgehende Aspekte ein. Explizit genannt wird zwar die Reaktionsfähigkeit (B03, Z. 94) und Wahrnehmung des Umfeldes (B04, Z. 105) aber es ist auch die Rede von \glqq der gesamten Fähigkeit, Auto zu fahren im Straßenverkehr\grqq{}. (B03, Z. 94)

\subsubsection*{Zuverlässiges Erkennen von Situationen}
Dass Fahrassistenzsysteme und autonome Fahrzeuge über die Fähigkeit verfügen, Verkehrssituationen und -teilnehmer insbesonbdere auch unter außergewöhnlichen Umständen korrekt zu erfassen, war für einen der Befragten eine konkrete Anforderung. (B02, Z. 70, 114)

\begin{quote}
  Also die müssen flexibel genug sein, alle Problematiken zu erkennen. Also zum Beispiel [...] ein Fahrrad mit [...] zwei Hinterrädern oder zwei Vorderrädern, eine andere Form, wird anders erkannt… [...] das System muss flexibel genug sein das zu erkennen, wenn auch mal ein Gerät ein bisschen anders aussieht als sonst. (B02, Z. 70)
\end{quote}

\subsubsection*{Vermitteln eines Sicherheitsgefühls}
Zwei Befragte nannten als Anforderung, dass Fahrassistenzsysteme und autonome Fahrzeuge dem Fahrer und auch den anderen Verkehrsteilnehmern ein Gefühl der Sicherheit vermitteln sollten. (B03, Z. 78; B04 Z. 39, 111) In diesem Zusammenhang klingt an, dass solche Systeme ihre Intention oft nicht klar machen; ein Befragter beschwert sich: \glqq kein Mensch fährt halt so\grqq{}. (B04, Z. 39)

\subsubsection*{Möglichkeit für menschliche Intervention}
Eine weitere von mehreren Befragten angeführte Anforderung war, dass der Mensch stets noch die Chance hat, manuell das Fahrgeschehen zu beeinflussen. (B02, Z. 82, 106; B06, Z. 36) Insbesondere war den Befragten dies wichtig, für den Fall, dass \glqq das Fahrzeug Dinge macht, die man nicht möchte\grqq{}. (B02, Z. 82)

\begin{quote}
  Was ich mir vorstellen könnte, dass man trotzdem noch eingreifen kann. (B06, Z. 36)
\end{quote}

Allen Befragten ging es hierbei in letzter Konsequenz hauptsächlich um das Vermeiden von Unfällen.

\subsubsection*{Geringe Ausfallquote}
Ein einzelner Befragter nannte als Anforderung, dass Fahrassistenzsysteme und autonome Fahrzeuge eine möglichst gerine Ausfallquote haben sollten.

\begin{quote}
  Oder besser gesagt geringe Fehlerrate, also keine Ausfall - geringe Ausfallquote. (B03, Z.52)
\end{quote}

\subsubsection*{Unabhängige Prüfstellen}
Einer der Befragten nannte die Anforderung, dass Systeme von unabhängigen, spezialisierten Prüfstellen abgenommen werden sollen. Als Vergleich wird der TÜV genannt.
\begin{quote}
  Ich finde, es sollte halt einfach sowas in Deutschland den TÜV, der sich damit dann hoffentlich auch auskennt sowas prüfen und testen [...]. (B04, Z. 107)
\end{quote}

\subsubsection*{Klar beschriebener Funktionsumfang}
Eine weitere genannte Anforderung ist ein klar beschriebener Funktionsumfang. Das System solle \glqq das tun, was draufsteht\grqq{}. (B03, Z. 52)


\subsection{Gefahren und Probleme}
In der Befragung wurde erhoben, welche möglichen Gefahren und Probleme die Kandidaten im Kontext von Fahrassistenzsystemen und autonomen Fahrzeugen wahrnehmen. Im Folgenden sind die genannten Punkte gelistet.

\subsubsection*{Wartung}
Ein Befragter äußerte Bedenken zu Gefahren, die durch mangelnde Wartung von autonomen oder unterstützenden Fahrsystemen ausgehen. Der Befragte äußerte, dass seiner Einschätzung nach viele Menschen ihre Fahrzeuge generell schon stark vernachlässigen würden. Er sieht ein Gefahrenpotenzial, falls entsprechende Systeme nicht ordnungsgemäß instandgehalten werden. Der Kandidat vergleicht das Gefahrenpotenzial mit Risiken, die durch die TÜV-Kontrollen abgedeckt werden und kommt zu der Einschätzung, dass sich das Niveau des Risikos auf der gleichen Stufe befindet. (B04, Z. 105)

\subsubsection*{Rechtliche Bedenken}
Die Hälfte der Befragten äußerte Bedenken, was rechtliche Fragen im Bezug auf Fahrassistenzsysteme und autonome Fahrzeuge angeht. Ganz allgemein formulierte es einer der Befragten: \glqq Viel Rechtliches. Noch. Weil viel da nicht geklärt ist und ich weiß nicht, wie schnell sich das entwickelt. Aber da muss man nachziehen.\grqq{} (B04, Z. 105)

Ein weiterer Aspekt, der genannt wurde ist der Datenschutz:
\begin{quote}
  Dass [...] Dritte wissen, wo ich mich gerade befinde und viele Zusatzinformationen über mein Auto haben, das würde mich stören. (B02, Z.76)
\end{quote}

Auf die Frage nach der Schuld im Falle eines Unfalls waren die Befragten unterschiedlicher Meinung. Als mögliche Schuldige bei einem Unfall nannten die Befragten:
\begin{itemize}
  \item Den Hersteller (B02, Z. 78)
  \item Den Programmierer der Software, falls die Software fehlerhaft war (B03, Z. 64, 100)
  \item Den Ingenieur des Fahrzeugs, falls die Konstruktion des Fahrzeugs fehlerhaft war (B03, Z. 100)
  \item Dem Fahrer, falls er seine Aufsichtsverantwortung bei einem Fahrassistenzsystem missachtet hat (B01, Z. 94; B04, Z. 75)
  \item Nicht dem Fahrer, falls das Fahrzeug autonom fährt (B04, Z. 75)
  \item Niemandem (B03, Z. 64, 100)
\end{itemize}

Die Befragten geben hier kein einheitliches Meinungsbild. Zu einer sehr differenzierten Einschätzung kommt einer der Befragten auf die Frage nach der Unfallschuld bei einem verunfallten autonomen Fahrzeug:
\begin{quote}
  Hm. Ich weiß nicht, wie - also nicht der Fahrer, meiner Meinung nach - aber ich weiß nicht, wie die Abgrenzung dann im Bereich ist von den verschiedenen Herstellern. Weil ich finde, dann kommt es stark darauf an, was versagt hat: War es Software, war es irgend ein Sensor... Warum, wenn ein Sensor? Liegt es daran, dass der kaputt war und der Fahrer sich nicht darum gekümmert hat, dann ist es doch wieder der Fahrer. Das sind halt viel zu viele kleine Faktoren als dass man das pauschalisieren könnte. (B04, Z. 115)
\end{quote}

\subsubsection*{Moralische Fragen}
Ebenfalls angeführt wurde die Problematik der moralischen Fragen, die im Kontext von autonomen Fahrzeugen und Fahrassistenzsystemen aufkommen. Ein Befragter nannte konkret das Beispiel von unvermeidlichen Unfällen, wo das Fahrzeug selbst eine Entscheidung zwischen zwei Übeln treffen muss. (B03, Z. 90)

\begin{quote}
  Wenn ein Unfall unvermeidbar ist, was macht das autonome Auto. Ja das ist die Moral. (B03, Z. 90)
\end{quote}

\subsubsection*{Überhöhtes Vertrauen in Technik}
Beinahe alle Befragten äußerten, dass Fahrer, die sich zu sehr auf ihr Fahrzeug verlassen ein Risiko darstellen. (B01, Z. 50; B02, Z. 66; B03, Z. 50; B04, Z. 57; B06, Z. 36)

\begin{quote}
  Ich glaub es kann gefährlich sein, wenn man sich zu sehr darauf verlässt und bei diesen Leuchten im Außenspiegel, die aufleuchten, wenn man überholt wird oder wenn ein Auto von hinten kommt, das man sich nur noch darauf verlässt und nicht mehr selber guckt. Und es kann immer sein das es mal nicht funktioniert und die Leute sich zu sehr darauf verlassen [...]. (B06, Z. 36)
\end{quote}

Darüber, wie sich das überhöhte Vertrauen konkret auswirken kann, hatten die Befragten unterschiedliche Einschätzungen:
\begin{itemize}
  \item Erhöhte Unfallgefahr (B01, Z. 50; B02, Z. 66; B03, Z. 50)
  \item Einschränkung/Verwirrung des Fahrers (B01, Z. 50)
  \item Erhöhte Wahrscheinlichkeit von Fahrfehlern (B03, Z. 50)
  \item Verleiten zu Unachtsamkeit (B04, Z.57)
\end{itemize}

\subsubsection*{Unachtsamkeit des Fahrers}
Als weitere Gefahr sehen die Befragten, dass Fahrer durch Fahrassistenzsysteme unachtsam im Straßenverkehr werden. Die Unachtsamkeit muss aber nicht notwendigerweise durch ein überhöhtes Vertrauen in die Assistenzsysteme eines Fahrzeugs bedingt sein, sondern kann auch grundsätzliche Unaufmerksamkeit des Fahrers bedeuten. Besonders hervorgehoben werden außergewöhnliche Situationen, in denen der Fahrer eingreifen müsste, es aber aufgrund von Unachtsamkeit nicht tut. (B04, Z. 57; B05, Z. 25)

Weiterhin genannt werden Assistenzsysteme, die aktiv vom Fahrer beachtet werden müssen, wie z.\,B. Toter-Winkel-Warner. Diese benötigen die Aufmerksamkeit des Fahrers, um wirksam zu sein. (B04, Z. 65)

\subsubsection*{Cyberattacken}
Zwei der Befragten sahen in der Tatsache, dass Fahrzeuge mit Fahrassistenzsystemen oder autonomen Fahrfunktionen häufig mit Netzwerk-Technolgien ausgestattet sind ein Gefahrenpotenzial für Cyberattacken. Explizit genannt wird die \glqq Ausnahmegefahr von Anschlägen\grqq{} (B03, Z.62) auf Grund der erhöhten Angreifbarkeit von zentral gesteuerten Systemen. (B01, Z. 66; B03, Z. 62, 92)

Neben der diffusen Gefahr des Hackerangriffs nannte ein Befragter konkrete Gefährdungsszenarien, die von Cyberattacken auf Fahrzeuge ausgehen: (B03, Z. 92)
\begin{itemize}
  \item Absichtliches Verursachen von Unfällen
  \item Entführen von Fahrzeugen aus der Ferne
\end{itemize}

\subsubsection*{Unzuverlässigkeit anderer (menschlicher) Verkehrsteilnehmer}
Weiterhin als Gefahr identifiziert werden die Unzulänglichkeiten nicht des autonomen bzw. unterstützen Autoverkehrs, sondern solche, die durch dritte Verkehrsteilnehmer entstehen. (B01, Z. 52-60, 88-92; B05.3 Z. 25)

\begin{quote}
  Ich glaube, dass sehr viele andere Autofahrer das Problem sind, und halt keine Standardsituationen wie Baustellen und Staus oder sowas. (B01, Z. 52)
\end{quote}

Hervorgehoben wurden von einer Person Fahrradfahrer, die sich über Verkehrsregeln hinwegsetzen. (B01, Z. 60)

\subsubsection*{Unerwartetes Verhalten des Autos}
Befragte schilderten außerdem, wie unerwartetes Verhalten ihres von einem Assistenzsystem geleiteten oder autonom fahrenden Vehikels Gefahre oder Probleme darstellen könnten. (B01, Z. 50; B02, Z. 106; B05.2, Z. 25; B06.2, Z. 42)

\begin{quote}
  Gerade wenn man in ein Parkhaus reinfährt, da macht der schon mal gern ne Vollbremsung vor so ner Schranke. (B01, Z. 50)
\end{quote}

Folgende Situationen wurden in diesem Zusammenhang von den Befragten genannt:
\begin{itemize}
  \item Aktiver Spurhalteassistent vom Fahrer vergessen: Das Auto lenkt gegen und \glqq überschreibt\grqq{} menschliche Aktion (B01, Z. 50)
  \item Zufahren auf Schranke (z.\,B.: im Parkhaus): Das Auto bremst stark und unerwartet (B01, Z. 50)
  \item Technischer Aussetzer: Der Fahrer ist nicht darauf vorbereitet, dass das Fahrassistenzsystem ausfällt (B05.2, Z. 25)
  \item Abbiegen des Autos: Von außen als Fahrradfahrer möglicherweise nicht erkennbar (B06.2, Z.42)
\end{itemize}

\subsubsection*{Erkennung von Situationen}
Vier der Teilnehmer nannten als mögliche Gefahr, dass Verkehrssituationen von den Fahrzeugen nicht korrekt erkannt werden.

\begin{quote}
  Wenn man zum Beispiel mit dem Spurhalteassistenten in die Baustelle reinfährt, erkennt der nicht, dass weiße Streifen aufhören und gelbe da sind, und versucht dann mitzulenken, das wird dann schwierig. (B01, Z. 44)
\end{quote}

Folgende Situationen wurden hierbei von den Befragten genannt:
\begin{itemize}
  \item Baustellen, bzw. der Wechsel von weißen zu gelben Fahrbahnmarkierungen (B01, Z. 44)
  \item Erkennung von Fahrzeugen in besonderen Lichtsituationen (B01, Z. 88)
  \item Parkhilfen, die Entfernungen falsch bemessen (B02, Z. 66)
  \item Fahrradfahrer, sowohl als false-positive (B04, Z. 67) als auch als false-negative (B04, Z. 65)
\end{itemize}

\subsubsection*{Ausfall der Sensorik}
Weiterhin wurde als potenzielle Gefahr der Ausfall der von den Assistenzsystemen oder autonomen Fahrzeugen benötigten Sensorik genannt. (B02, Z. 66, 106; B06.3, Z. 18, 28)

\begin{quote}
  Also dass die Sensorik zum Beispiel bei einem Abstandshalter spinnt, kaputtgeht, und dann nicht mehr funktioniert. (B02, Z. 66)
\end{quote}

In diesem Zusammenhang wurden von einer Person mehrfach Vergleiche zu Alltagstechnologien wie Handy oder Laptop und deren in ihrer Erfahrung hohen Ausfallquote gezogen. (B06.3, Z. 18, 28)


\subsection{Vertrauen}
Während der Interviews äußerten sich mehrere Befragten in Bezug auf Vertrauen. Diese Äußerungen werden hier strukturiert wiedergegeben.

\subsubsection*{Institutionen in Deutschland}
Ein Befrater äußerte, dass seiner Einschätzung nach insbesondere in Deutschland ausschließlich funktionierende Fahrzeuge für den Straßenverkehr zugelassen sind:

\begin{quote}
  Das wird - vor allem, gehen wir jetzt mal wir leben ja in Deutschland, das wird ausgiebig getestet vermutlich. Ich denke, das wird funktionieren, sonst würde das nicht einfach so auf die Straße gelassen werden. (B04, Z. 51)
\end{quote}

\subsubsection*{Historie als Vertrauensfaktor}
Für das Vertrauen zu Automobilherstellern und deren Fahrzeuge war für einen der Befragten das Verhalten dieses Unternehmens in der Vergangenheit besonders wichtig. (B03, Z. 48, 88)

\begin{quote}
  Ich wüsste jetzt nicht, dass irgendwer da grob Unfähigkeiten gibt oder gezeigt hat, dass er grob unfähig in dem Bereich ist. (B03, Z. 88)
\end{quote}

Explizit genannt wurde in diesem Zusammenhang auch der Dieselskandal. (B03, Z. 48)

\subsubsection*{Vertrauen in Automarken}
Auf die Frage nach Miss- oder Vertrauen in einzelne Automarken gaben die Interviewpartner verschiedene Antworten.

Einer der Befragten nannte Tesla als vertrauenswürdige Automarke und begründete diese Entscheidung mit langen unfallfreien Fahrzeiten und einer von ihm positiv Bewerteten Krisenkommunikation des Geschäftsführers von Tesla. Als Vergleichsobjekt zog der Befragte Ford heran; Er würde Tesla mehr vertrauen als zum Beispiel Ford. (B01, 82ff)

Weiterhin wurden VW und Audi als nicht vertrauenswürdige Automarken genannt, Grund hierfür sei der Dieselskandal. (B03, Z.42; B05.2, Z. 15) Einer der Befragten ging zusätzlich ins Detail: Obwohl er den Unternehmen grundsätzlich gute Intentionen unterstellt, herrscht Misstrauen; Sowohl was die Qualität der Produkte in Punkto Fehlerfreiheit angeht, als auch was die Selbstdarstellung des Unternehmens in Marketing und Öffentlichkeitsarbeit angeht. Abwiegelnd äußert der Befragte außerdem, dass die Großunternehmen auch monetäre Interessen verfolgen. (B03, Z. 42ff)

Von allgemein guten Erfahrungen mit den Automarken VW (mit Einschränkung des Abgasskandals), BMW, Audi, Skoda und Volvo berichtet ein weiterer Befragter. (B05.2, Z.15) Auf Nachfrage nennt er folgende Kriterien für seine Bewertung von Automarken (B05.2, Z. 17):
\begin{itemize}
  \item Fahrkomfort
  \item Sicherheit
  \item Fahrspaß
  \item Erfüllen des gewünschten Funktionsumfangs
\end{itemize}

Davon abgesehen gab es eine wesentlich abstraktere Einschätzung: Einer der Befragten beurteilte die Qualität der Fahrassistenzsysteme als \glqq auf dem ungefähr gleichen Niveau\grqq{} (B04, Z. 49) fügte aber im weiteren Verlauf des Interviews hinzu:

\begin{quote}
  Ja das auf sone Automarkengeschichte herunterzupressen find ich schwierig. In Automarken geht auch immer viel persönlicher Geschmack und sowas ein. [...] Ich weiß auch nicht ob jeder Autohersteller das komplett selber entwickelt, deshalb ist das schwierig, das mit der Marke in Verbindung zu setzen [...]. (B04, Z. 103)
\end{quote}

\subsubsection*{Vertrauen durch Erfahrungswerte}
Mehrere Befragte vermissen persönliche Erfahrungen mit Fahrassistenzsystemen oder autonomen Fahrzeugen, um sich eine bessere Meinung bilden zu können. Hierbei ging es den Befragten um eine Befriedigung der persönlichen Neugierde (B04, Z. 91ff), Vertrauensbildung (B05.3 Z.22; B06.3, Z. 24ff) oder allgemeine Erweiterung des Erfahrungsschatzes (B06.2, Z. 32)

\begin{quote}
  Ich glaube es würde helfen, wenn die Sachen mehr Präsenz im Alltag bekommen würde, man bekommt hier einfach nichts mit bzw. ich kriege nichts mit. Wenn es einfach ein bisschen präsenter werden würde. Dann sieht man das es funktioniert und es gut funktioniert. (B06.3, Z. 24)
\end{quote}

\subsubsection*{Vertrauen durch Transparenz}
Ein weiterer Aspekt, der für Vertrauen genannt wurde, war Transparenz. Dies hieß für die Befragten Unterschiedliches:
\begin{itemize}
  \item Aufbereitete Rohdaten von Tests (B02, Z. 60ff)
  \item Aufrichtige Kommunikation über Produkte (\glqq Vernünftig produzieren und das abliefern, was sie auch draufschreiben\grqq{} (B03, Z. 84))
  \item Offene Kommunikation über Wissensstand und Fortschritt des Unternehmens (B04, Z. 99)
  \item Angemessene kommunikative Lösungen für Restrisiken (B04, Z. 105)
\end{itemize}

\subsection{Abgeben von Fahraufgaben}
In den Gesprächen wurden Situationen erfragt, in denen die Befragten gerne Fahraufgaben abgeben würden oder bereits auf Fahraufgaben verzichten. Weiterhin wurden Argumente für und gegen das Abgeben dieser Fahraufgaben gesammelt.

\subsubsection*{Stadtstrecken}
Bereitschaft zum abgeben von Fahraufgaben herrscht bei zwei der Befragten in Städten (B02, Z. 20; B04, Z. 95), einer der beiden fügt hinzu, hebt fremde bzw. unbekannte Städte hervor und nennt Überforderung mit Ampeln und Abbiegerspuren als Befründung.

\begin{quote}
  Und in unbekannten Städten, also in Städten wo ich nicht weiß, wie man normalerweise fährt. Da bin ich schnell überfordert mit Ampeln, Abbiegerspuren etc. (B04, Z. 95)
\end{quote}

\subsubsection*{Alkoholkonsum}
Einer der Befragten nannte als Situation, in der er das Fahren vermeidet den Zustand der Trunkenheit oder des Verkatert-Seins.

\begin{quote}
  Wenn ich verkatert bin. Dann fahr ich danach eigentlich sehr ungern Auto. (B03, Z. 14)
\end{quote}

Es kann mit einiger Wahrscheinlichkeit davon ausgegangen werden, dass dies für die übrigen Befragten so selbstverständlich war, dass sie es nicht gesondert erwähnt haben.

\subsubsection*{Bremsen}
Ein Befragter äußerte, dass er gerne das Bremsen an einen Bremsassistenten abgeben würde, um Auffahrunfälle zu vermeiden.

\begin{quote}
  Also ich kann mir vorstellen, zum Beispiel ein Bremssystem. [...] Ein Abstandsmesser der eingreift, wenn du gedanklich kurz abwesend bist. Damit können Auffahrunfälle nicht mehr passieren. (B05.1, Z. 16)
\end{quote}

\subsubsection*{Autobahnfahrten}
Mehrfach genannt wurden lange Fahrten auf der Autobahn, bei denen Befragte gerne Teile des Fahrens an Assistenzsysteme abgeben würden. (B01, Z. 38; B02, Z. 28ff; B03, Z. 22, 62; B04, Z. 23, 37, 95; B05.1, Z. 10; B06.1, Z. 20; B06.2, Z. 58; B06.3, Z. 10)

Folgende Fahraufgaben sollen bei Autobahnfahrten abgegeben werden:
\begin{itemize}
  \item Abstand halten (B02, Z. 28; B04, Z. 23)
  \item Geschwindigkeit halten (B02, Z. 28; B06.1, Z. 20)
  \item Stop-and-Go im Stau (B03, Z. 62; B04, Z. 23; B05.1, Z. 10B06.1 Z. 20)
  \item Bilden einer Rettungsgasse (B06.2 Z. 58)
\end{itemize}

Folgende Gründe werden zum Abgeben der Fahraufgaben bei Autobahnfarten genannt:
\begin{itemize}
  \item Konzentration über lange Zeit (B02, Z. 8; B06.3, Z. 10)
  \item Ermüdung auf langen Strecken (B04, Z. 37)
  \item Fahraufgaben nervenaufreibend (B03, Z. 62; B05.1, Z. 10; B06.1, Z. 20)
  \item Könnte Stau reduzieren (B04, Z. 23)
  \item Menschliche Autofahrer nicht fähig genug (B06.2, Z. 58)
\end{itemize}

\subsubsection*{Einparken}
Ein weiterer Aspekt, der in einem Gespräch genannt wurde ist das Einparken.

\begin{quote}
  Ja, also parken könnte ich mir auch vorstellen, dass mir das abgenommen wird. (B02, Z. 34)
\end{quote}

Der Befragte beschrieb Parken als \glqq Situation, die ich sehr schlecht finde\grqq{} (B02, Z. 20) und wünscht sich entsprechend ein System, das ihn bei dieser Fahraufgabe unterstützt.

\subsubsection*{Negative Grundhaltung zum Autofahren}
Einer der Befragten nannte als möglichen Grund für das Abgeben von Fahraufgaben, dass Personen möglicherweise grundsätzlich nicht gerne mit dem Auto fahren. (B03, Z.22)

\subsubsection*{Körperliche Einschränkungen}
Körperliche Einschränkungen können auch ein Grund dafür sein, dass Personen Fahraugaben abgeben wollen. So nannten Befragte das Alter des Fahrers oder eingeschränktes Sehvermögen. (B03, Z. 22) Darüber hinaus äußerte ein Interviewpartner, dass er sich nach starker körperlicher Anstrengung manchmal wünscht, das manuelle Schalten aufzugeben und seinem Fahrzeug zu überlassen: (B01, Z. 18)

\begin{quote}
  Also, gerade nach nem Marathon oder Triathlon sind die Beine manchmal etwas schwer. Da wünscht man sich dann doch ne Automatik.
\end{quote}


\subsection{Behalten von Fahraufgaben}
Ebenfalls erhoben wurde, welche Argumente es für Autofahrer gibt, Fahraugaben nicht abzugeben sondern zu behalten.

\subsubsection*{Mangelndes Vertrauen in Technik}
Einer der Befragten nannte mangelndes Vertrauen als möglichen Grund für Fahrer, Fahraufgaben lieber selbst zu erledigen. Dabei bezieht er Vertrauen sowohl auf die Qualität des Systems als auch auf die Sicherheit. (B04, Z. 25, 39)

\begin{quote}
  Weil ich glaube, die meisten haben nicht so das Vertrauen und würden das nicht gerne abgeben wollen. Und denken auch, sie können es besser [...]. (B04, Z. 25)
\end{quote}

\subsubsection*{Unwissen über Möglichkeiten}
Einer der Befragten antwortete auf die Fragen danach, ob er Teile des Fahrens abgeben wollen würde mit:

\begin{quote}
  Ne. Also. Ich wüsste jetzt nicht, welche. (B03, Z. 20)
\end{quote}

\subsubsection*{Selbstbestimmtheit}
Ein weiteres Argument dafür, Fahraufgaben zu behalten scheint die Selbstbestimmtheit des Autofahrers zu sein. Hierbei spielt zum Beispiel das bewusste Übertreten von Geschwindigkeitsbegrenzungen eine Rolle, die sonst vom Assistenten oder autonomen Fahrzeug penibel eingehalten würden. (B02, Z. 36; B04, Z. 37)

\subsubsection*{Freude am Autofahren}
Mehrere der Befragten nannten Fahrspaß bzw. freude am Autofahren als Hindernis dafür, Fahrassistenzsysteme oder autonome Fahrzeuge zu verwenden. (B01, Z. 18, 30; B02, Z. 36; B04, Z. 95; B06.1, Z. 22)

Als besonderer Spaßfaktor wurde mehrfach schnelles Autofahren genannt. (B02, Z. 36; B04, Z. 95)

\begin{quote}
  Weil es mir doch manchmal zu viel Spaß macht, selber zu fahren. Ich glaube, so Stadtstrecken oder so könnte ich gerne abgeben, weil das ist jetzt nicht das Spaßigste, Unterhaltsamste aber so schöne Bundesstraßen oder mal auf der Autobahn, das fahre ich dann doch lieber selbst. Es sei denn, es kommt Stau, dann kann ichs auch wieder abgeben. (B04, Z. 95)
\end{quote}


\clearpage
\section{Diskussion}

\clearpage
\section{Methodenreflexion}

\clearpage
\section{Fazit}

\clearpage
\pagenumbering{Roman}

\clearpage
\appendix
\pagestyle{appendix}

\label{att:Anhang1}
\fakesection{Anhang1}
% \includepdf[pages={1},scale=0.9,pagecommand={\section{Anhang}}]{}

\section{Literaturverzeichnis}
\pagestyle{MRstyle}
\setcounter{biburllcpenalty}{7000}
\setcounter{biburlucpenalty}{8000}
\printbibliography
\pagebreak

\pagestyle{plain}
\pagenumbering{gobble}
%\includepdf[pagecommand={}]{Eidesstattliche_Versicherung.pdf}

\end{document}
