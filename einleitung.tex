\section{Einleitung}

Der Mensch strebt nach stetigem Fortschritt. Computer werden immer schneller, Wolkenkratzer immer höher und Flugzeuge immer komfortabler. Auch anstrengende oder eintönige Arbeiten, etwa am Fließband, werden inzwischen von Maschinen verrichtet, die stetig verbessert werden (vgl. \cite{makino1994new}).

Dieser Fortschritt macht auch nicht vor Autos halt. Seit der Erfindung des ersten Automobils wurde zwar viel an Motoren, Navigationssystemen und eingebauten Freisprechanlagen erfunden, das Prinzip blieb jedoch größtenteils unverändert: Ein menschlicher Fahrer steuerte das Fahrzeug und all seine Funktionen. In den vergangenen Jahren wurde immer mehr Forschung unternommen, um genau dieses Verhältnis von Mensch und Auto aufzubrechen (vgl. \cite{wei2013towards}). Nicht nur Autobauer wie Volvo und Mercedes arbeiten an sogenannten \emph{autonomen Fahrzeugen}, die selbstständig und ohne Hilfe durch einen Menschen durch den Straßenverkehr navigieren können. Auch Tech-Firmen wie Apple, Google und Uber investieren in diese zukunftsträchtige Technologie.

Doch die Vorstellung, dass Autos in Zukunft ohne Eingriff des Fahrers fahren, gefällt nicht allen. In den Onlinekommentaren unter Artikeln, in denen Unfälle von solchen autonomen Fahrzeugen thematisiert werden, herrscht allgemeine Angst und unzufriedenheit. Für viele der Kommentierenden sind autonome Fahrzeuge gefährlicher als Menschen, bauen mehr Unfälle und können sich nie mit menschlichen Fahrern messen. Doch sind autonome Fahrzeuge tatsächlich derart gefährlich -- und Menschen derart so Fahrer? Unfallstatistiken deuten tatsächlich darauf hin, dass objektiv betrachtet autonome Fahrzeuge weniger Fehler machen (und damit weniger Unfälle verursachen) als Menschen (vgl. \cite{singh2015critical}). Die allgemeine Gefühlslage im Internet scheint diesen objektiven Zahlen allerdings nicht zu trauen. Hersteller solcher autonomer Fahrzeuge stehen also vor der Frage, wie sie das augenscheinlich unbegründete, negative Image von autonomen Fahrzeugen aufbessern können.

Im Rahmen dieser Hausarbeit soll das Bild, das von autonomen Fahrzeugen herrscht, näher untersucht werden. Hierzu wurde zunächst eine Web-Studie durchgeführt, in der stichpunktartig Kommentare in Onlineartikeln und der Mikrobloggingplattform Twitter ausgewertet wurden. Auf Grundlage dieser Ergebnisse wurde ein erstes Kategoriensystem und ein Interviewleitfaden erstellt, mit dem einige der Kernaussagen aus der Webstudie in qualitativen Face to Face-Interviews näher untersucht werden sollten. Zuletzt wurden diese Interviews ausgewertet und die Ergebnisse diskutiert.