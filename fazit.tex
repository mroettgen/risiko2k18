\section{Fazit}

Die durchgeführte Studie konnte zwar vor allem wegen der kleinen Stichprobe keine wirklich belastbaren Ergebnisse zutage bringen, bietet aber interessante Ansatzpunkte für weitere Forschung.

Zum einen wich die grundsätzliche Einstellung der Interviewpartner deutlich vom sehr negativ geprägten Diskurs im Internet ab. Hier wäre eine größer angelegte, quantitative Studie interessant, um ein repräsentativeres Bild der öffentlichen Wahrnehmung von autonomen Fahrzeugen zu gewinnen. Viele waren durchaus reflektiert und sagten, dass es nur in der öffentlichen Wahrnehmung so wirke, als würden autonome Fahrzeuge mehr Unfälle bauen als menschliche Fahrer (vgl. B04, Z. 113). 

Zudem hatten die Befragten noch keine eigene Erfahrung mit autonomen Fahrzeugen machen können -- entsprechend konnten sie auch nicht mit Sicherheit sagen, ob sie sich in einem solchen Fahrzeug sicher fühlen würden oder nicht, sondern konnten lediglich eine Annahme dazu geben. Hier wäre es für zukünftige Studien interessant auch Probanden zu befragen, die bereits Erfahrungen mit autonomen Fahrzeugen sammeln konnten.

Einige der Befragten nannten konkrete Vorschläge, wie Vertrauen gewonnen werden könnte -- etwa durch die Veröffentlichung von Testberichten oder einer unabhängigen Prüfstelle. Hier könnte weitere Forschung ansetzen, die beispielsweise die Anforderungen an solch eine Prüfstelle herausfinden könnte.